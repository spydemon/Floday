\subsection{L'exécution}

\subsubsection{Les différentes étapes}
\existstill{1.0.0}

Voici les différentes étapes par lesquelles nous passons pour le déploiement d'un hôte~:
La figure \ref{1.4_algorithme_complet} illustre à peu prêt le fonctionnement via un genre d'\emph{UML}.
Ce schéma pourrait être plus représentatif, mais je me suis promis de ne pas passer plus de 7h dessus.

\begin{figure}
	\centerfloat
	\caption{Illustration du processus de définition de configuration et de déploiement d'applications.}
	\label{1.4_definition_runlist}
	\begin{tikzpicture}[
  boxstyle/.style={%
    draw=gray!70
  },
  varrow/.style={%
    rectangle,
    minimum height=2cm,
    minimum width=0.5cm,
    draw=gray!70
  },
  harrow/.style={%
    right,
    rectangle,
    minimum width=5.5cm,
    minimum height=0.5cm,
    draw=gray!70
  }
]

	%%%%%%%%%%%
	% Runfile %
	%%%%%%%%%%%
	\draw[boxstyle] (-.75, 2.5) rectangle (4.75, -2);
	\node[text width=3.75cm, right, text centered] at (0, 2) {Runfile};
	\node[text width=3.75cm, right, text justified] at (0, 0) {%
		\begin{lstlisting}[basicstyle=\scriptsize]
host:
  spyzone:[...]
    applications:
      web:[...]
        parameters:
          type: lighttpd
          ipv4: 10.0.3.5
          ssl: spyzone.fr
        applications:
          my_blog:[...]
          mum_blog:[...]
		\end{lstlisting}
	};

	%%%%%%%%%%%%%%%%%%%%
	% riuk-core config %
	%%%%%%%%%%%%%%%%%%%%
	\node[text width=6.25, right, text justified] at (6, 5) {%
		\begin{lstlisting}[basicstyle=\scriptsize]
parameters:
  gateway:
    mandatory: true
      pattern: [...]
    value: 10.0.0.1
  ipv4:
    mandatory: true
      pattern: [...]
  template:
    mandatory: true
    value: flod...
setups:
  networking:
    exec: r/c/c/s/net.pl
    priority: 1-10
  ssh:
    exec: r/c/c/s/ssh.pl
end_setups:
  iptables:
    exec: r/c/c/e/iptbl.pl
    priority: 1-10
		\end{lstlisting}
	};

	%%%%%%%%%%%%%%%%%%%%%%%%
	% riuk-lighttpd config %
	%%%%%%%%%%%%%%%%%%%%%%%%
	\node[text width=6.25cm, right, text justified] at (6, 0) {%
		\begin{lstlisting}[basicstyle=\scriptsize]
inherit:
  - riuk-core
setups:
  lighttpd_install:
    exec: r/c/l/s/ltpd.pl
    priority: 2-10
parameters:
  ssl:
    mandatory: false
		\end{lstlisting}
	};

	%%%%%%%%%%%%%%%
	% Surcharges %
	%%%%%%%%%%%%%%%
	\node[text width=4cm, right, text centered, draw, boxstyle] at (6, -3) {Surcharges du runfile};

	%%%%%%%%%%%
	% Runlist %
	%%%%%%%%%%%
	\node[text width=6.25, text justified] at (11.5, 6) {%
		\begin{lstlisting}[basicstyle=\scriptsize]
host:
  spyzone:[...]
    applications:
      web:
        parameters:
          gateway: 10.0.0.1
          ipv4: 10.0.3.5
          ssl: spyzone.fr
          template: flodayalpine
          type: lighttpd
        setups:
          1-10: r/c/c/s/net.pl
          1-20: r/c/c/s/ssh.pl
          2-10: r/c/l/s/ltpd.pl
        end_setups:
          2-10: r/c/l/s/iptbl.pl
        applications:
          my_blog: [...]
          mum_blog: [...]
		\end{lstlisting}
	};

	\draw[decorate, boxstyle, decoration={brace, amplitude=10pt}] (17, 8.5) -- (17, -2) node {};

	%%%%%%%%%%%%%%%%%%%%%%
	% Flèche horisontale %
	%%%%%%%%%%%%%%%%%%%%%%
	\node at (8.64, 10.5) {Définition de l'application};
	\node[single arrow, draw=gray!70, minimum height=17.277cm, minimum width=1.5cm, right] at (0,10) {};
	\node[harrow](h_arrow_1) at (0, 10) {1};
	\node[harrow](h_arrow_2) [right=0cm of h_arrow_1] {2};
	\node[harrow](h_arrow_3) [right=0cm of h_arrow_2] {3};

	%%%%%%%%%%%%%%%%%%%%
	% Flèche verticale %
	%%%%%%%%%%%%%%%%%%%%
	\node[rotate=-90] at (18.5, 3.1) {Déploiement de l'application};
	\node[single arrow, draw=gray!70, minimum height=10.8cm, minimum width=1.5cm, rotate=-90, right] at (18, 8.5) {};
	\node[varrow](v_arrow_1) at (18, 7.5) {1};
	\node[varrow](v_arrow_2) [below=0cm of v_arrow_1] {2};
	\node[varrow](v_arrow_3) [below=0cm of v_arrow_2] {3};
	\node[varrow](v_arrow_4) [below=0cm of v_arrow_3] {4};
	\node[varrow](v_arrow_5) [below=0cm of v_arrow_4] {5};
\end{tikzpicture}


\end{figure}

\begin{figure}
	\centerfloat
	\caption{Algorithme complet d'un dépoilement}
	\label{1.4_algorithme_complet}
	\begin{tikzpicture}[
  boxstyle/.style={%
    draw=gray!70
  },
  varrow/.style={%
    rectangle,
    minimum height=2cm,
    minimum width=0.5cm,
    draw=gray!70
  },
  harrow/.style={%
    right,
    rectangle,
    minimum width=5.5cm,
    minimum height=0.5cm,
    draw=gray!70
  }
]

	%\umlstateinitial[x=0, y=0, name=initial]
	\begin{umlstate}[x=2.70, y=-4, name=defapp, width=20ex]{Définition de la runlist}
		\umlstateinitial[x=-3, y=0, name=init];
		\umlbasicstate[x=0, y=0, name=runfile]{Ouverture du runfile};
		\umltrans{init}{runfile};
		\umlstatedecision[y=-2, name=stillapp]{Applications restantes dans le host à déployer~?};
		\umltrans[pos=0.5, arg=Applications restantes à parser ?]{runfile}{stillapp};
		\umlbasicstate[x=0, y=-4, name=defineapp]{Définition de l'application};
		\umltrans[recursive=180|150|6cm, arg=Définition des contraints récursivement, pos=2.5, recursive direction=left to top]{defineapp}{defineapp};
		\umltrans[pos=0.1, arg=Oui]{stillapp}{defineapp};
		\umlbasicstate[x=5, y=-4, name=validaterunfile]{Validation du runfile};
		\umlHVtrans[pos=0.1, arg=Non]{stillapp}{validaterunfile};
		\umlstatefinal[x=.5, y=-6, name=validationfailed];
		\umlstatefinal[x=9.5, y=-6, name=validationsucess];
		\umlVHtrans[pos=1.5, arg=Échec de la validation]{validaterunfile}{validationfailed};
		\umlVHtrans[pos=1.5, arg=Validation effectuée]{validaterunfile}{validationsucess};
		\umltrans{defineapp}{validaterunfile}
	\end{umlstate}
	%\umlHVtrans{initial}{defapp}
	\begin{umlstate}[x=4.4, y=-14, name=runapp, width=200cm]{Déploiement des applications}
		\umlstateinitial[x=0, y=0, name=initialdeploy]
		\umlstatedecision[y=-1.5, name=stilldeplapp]{};
		\umltrans[pos=0.5, arg=Applications restantes à déployer ?]{initialdeploy}{stilldeplapp};
		\umlstatedecision[y=-3, name=appallreadyexisting];
		\umltrans[pos=0.5, arg=L'application existe déjà ?]{stilldeplapp}{appallreadyexisting};
		%\umlbasicstate[x=-3, y=-5, name=appdestroy]{Destruction application};
		\umlbasicstate[x=-3, y=-5, name=appdestroy]{Destruction application}{Infos};
		\umlHVtrans[pos=0.1, arg=Oui]{appallreadyexisting}{appdestroy};
		\umlstatedecision[y=-6, name=endappallreadyexisting]{};
		\umlVHtrans{appdestroy}{endappallreadyexisting};
		\umltrans[pos=0.1, arg=Non]{appallreadyexisting}{endappallreadyexisting};
		\umlbasicstate[x=0, y=-8, name=predeploy]{Pré-installation};
		\umltrans{endappallreadyexisting}{predeploy};
		\umlstatedecision[y=-10, name=haschildren]{};
		\umltrans[pos=0.5, arg=À encore des contraints à déployer ?]{predeploy}{haschildren};
		%\umlHVHtrans[pos=0.1, arg=Oui, arm1=-6cm]{haschildren}{stilldeplapp};
		\umlHVHtrans[pos=0.1, arg=Oui, arm1=-6cm]{haschildren}{initialdeploy};
		\umlbasicstate[x=0, y=-12, name=postdeploy]{Post-installation};
		\umltrans[pos=0.1, arg=Non]{haschildren}{postdeploy};
		\umlstatefinal[x=6, y=-11.75, name=enddeploy];
		\umltrans{postdeploy}{enddeploy};
	\end{umlstate}
	\umltrans{defapp}{runapp};
\end{tikzpicture}

\end{figure}

\begin{description}
	\item[Définition de la \gls{runlist}]
		Cette étape permet d'obtenir un tableau multi-dimentionnel clair de tous les éléments structurants nécessaires au déploiement.
		Il peut lui-même se séparer en trois étapes~:
		\begin{description}
			\item[Ouverture du \gls{runfile}]
				Nous récupérons le \emph{runfile} depuis le fichier de configuration.
				Il est ensuite parcouru pour trouver les \glspl{chemin_conteneur} à déployer.
			\item[Définition des applications]
				En fonction des \glspl{chemin_conteneur} obtenus, on cherche à présent les fichiers \gls{definition} des conteneurs utilisés.
				Les éventuels héritages sont gérés à ce moment.
				À présent, sont mergées les valeurs propres à l'application courante à déployer.
				Il s'agit des \glspl{param_applicatifs} présents au niveau du runfile.
			\item[Validation de la runlist]
				Nous pouvons à présent valider la conformité de la \gls{runlist}.
				Si quelque chose ne correspond pas, le déploiement est annulé avant de faire réellement quoi que ce soit.
		\end{description}
		La figure~\ref{1.4_definition_runlist} illustre ces principes de façon plus visuelle.
	\item[Déploiement des applications]
		Pour chaque applications présentes dans l'hôte en cours de déploiement au niveau du runfile, on effectue les actions suivantes~:
			\begin{description}
				\item[Destruction de l'application]
					S'il s'agit d'un redéploiement, c'est-à-dire qu'une application avec un chemain similaire est déjà existante, la première étape consistera en la destruction de celui-ci.
					C'est à cette étape que nous exécuterons les hooks \path{lxc-destroy-before} et \path{lxc-destroy-after}. Bien-entenu, entre ces deux hooks, le précédent conteneur servant à accueillir l'ancienne version de l'application sera détruit.
				\item[Pré-installation]
					On commencera par exécuter les scripts présents au niveau du hook \path{lxc-deploy-before}.
					Ensuite le conteneur \emph{LXC} sera déployé via le template défini (par défaut \emph{lxc-flodayalpine}). Nous jouerons ensuite les hooks \path{lxc-deploy-after}.
					Finalement, c'est l'ensemble des scripts \emph{setup} de l'application en cours de déploiement qui seront exécutés.
				\item[Déploiement des sous-applications]
					Les \glspl{sous-application}es sont ensuite intégrallement déployées (elles passent donc elles aussi par l'intégralité des étapes de la partie \emph{Déploiement des applications} de cette liste.
				\item[Post-installation]
					Une fois toutes les sous-applications correctement déployés, nous exécutons les scripts présents dans le nœud \emph{end\_setup}.
			\end{description}
\end{description}
Et voilà, on se retrouve à terme avec un hôte correctement déployé.

\subsubsection{La gestion des logs}
\existstill{1.0.0}

Une fois notre \emph{runfile} et les applications qui s'y réfèrent de définies, il ne reste plus qu'à amorcer le tout et permettre à \emph{Floday} d'effectuer sa principale raison d'être~!
Pour cela nous, n'avons qu'à lancer la commande \path{floday} avec l'option \path{--host} permettant de définir quel \gls{hote} du runfile nous voulons déployer.
Des logs d'exécution seront ensuite affichés sur \emph{STDOUT} ainsi que via \emph{syslog}.

\paragraph{Gestion du niveau des logs}
Il n'est malheureusement pas possible actuellement de définir soi-même le niveau de verbosité à avoir sans toucher au code de l'application, mais ce problème sera peut-être résolu un jour%
\footnote{Un ticket a été créé sur le bugtracker pour rendre la gestion du niveau de log plus pratique~: \url{https://dev.spyzone.fr/floday/ticket/29}}.

Actuellement, pour le modifier, il faudra donc changer la valeur de la ligne 18 du fichier \emph{floday.pl}.
Les niveaux disponibles sont ceux habituellement en vigeur sur n'importe quel système \emph{UNIX}.
Pour plus de détails, voir le hash \emph{\%SYSLOG\_PRIORITY\_MAPPER} dans le fichier \emph{Floday/Helper/Logging.pm}.

\begin{lstlisting}[caption={config.yml}, label=fig_1.4_floday, basicstyle=\tiny, xleftmargin=-3cm]
# ./floday.pl --host websites
[WARN]                 Floday::Deploy:   Deploying websites host
[WARN]                 Floday::Deploy:    Start running setups scripts.
[INFO]                 Floday::Deploy:     Running script: /etc/floday/containers/jaxe/setups/iptables_flush.pl
[INFO]                 Floday::Deploy:     Running script: /etc/floday/containers/jaxe/setups/dns.pl
[INFO]                 Floday::Deploy:     Running script: /etc/floday/containers/jaxe/setups/ssh.pl
[INFO]                 Floday::Deploy:     Running script: /etc/floday/containers/jaxe/setups/network.pl
[INFO]                 Floday::Deploy:     Running script: /etc/floday/containers/jaxe/children/perl/setups/cpan.pl
[INFO]                 Floday::Deploy:     Running script: /etc/floday/containers/jaxe/children/perl/setups/updater.pl
 [ERR]        Floday::Lib::Linux::LXC:      lxc-attach: failed to get the init pid
[INFO]                 Floday::Deploy:     Running script: /etc/floday/containers/jaxe/setups/updater.pl
[INFO]                 Floday::Deploy:     Running script: /etc/floday/containers/jaxe/setups/alpine_preparation.pl
[INFO]                 Floday::Deploy:     Running script: /etc/floday/containers/jaxe/setups/debian_preparation.pl
[WARN]                 Floday::Deploy:    End running setups scripts.
[WARN]                 Floday::Deploy:    Start deployment of websites applications.
[WARN]                 Floday::Deploy:     Launching website-web_application application.
[WARN]        Floday::Lib::Linux::LXC:      Start pre destruction hooks.
[INFO]        Floday::Lib::Linux::LXC:       Running script: jaxe/children/core/hooks/lxc_destroy_before/btrfs_management.pl
[WARN]        Floday::Lib::Linux::LXC:      End pre destruction hooks.
[WARN]        Floday::Lib::Linux::LXC:      Start LXC container website-web_application destruction.
[WARN]        Floday::Lib::Linux::LXC:      End LXC container destruction.
[WARN]        Floday::Lib::Linux::LXC:      Start post destroy hook.
[WARN]        Floday::Lib::Linux::LXC:      End post destroy hook.
[WARN]        Floday::Lib::Linux::LXC:      Start pre deployment hooks.
[INFO]        Floday::Lib::Linux::LXC:       Running script: jaxe/children/core/hooks/lxc_deploy_before/btrfs_management.pl
[WARN]        Floday::Lib::Linux::LXC:      End pre deployment hook.
[WARN]        Floday::Lib::Linux::LXC:      Start deploying LXC website-web_application  container.
[WARN]        Floday::Lib::Linux::LXC:       End deploying LXC container.
[WARN]        Floday::Lib::Linux::LXC:      Start post deployment hook.
[WARN]        Floday::Lib::Linux::LXC:      End post deployment hook.
[WARN]                 Floday::Deploy:      Start running setups scripts.
[INFO]                 Floday::Deploy:       Running script: /etc/floday/containers/jaxe/children/core/setups/networking.pl
[INFO]                 Floday::Deploy:       Running script: /etc/floday/containers/jaxe/children/core/setups/cleaning.pl
[INFO]                 Floday::Deploy:       Running script: /etc/floday/containers/jaxe/children/core/setups/updater.pl
[INFO]                 Floday::Deploy:       Running script: /etc/floday/containers/jaxe/children/www/setups/lighttpd.pl
[WARN]                 Floday::Deploy:      End running setups scripts.
[WARN]                 Floday::Deploy:      Launching website-web_application-galek application.
[WARN]        Floday::Lib::Linux::LXC:       Start pre destruction hooks.
[INFO]        Floday::Lib::Linux::LXC:        Running script: jaxe/children/core/hooks/lxc_destroy_before/btrfs_management.pl
[WARN]        Floday::Lib::Linux::LXC:       End pre destruction hooks.
[WARN]        Floday::Lib::Linux::LXC:       Start LXC container spyzone-web-galek destruction.
[WARN]        Floday::Lib::Linux::LXC:       End LXC container destruction.
[WARN]        Floday::Lib::Linux::LXC:       Start post destroy hook.
[WARN]        Floday::Lib::Linux::LXC:       End post destroy hook.
[WARN]        Floday::Lib::Linux::LXC:       Start pre deployment hooks.
[INFO]        Floday::Lib::Linux::LXC:        Running script: jaxe/children/core/hooks/lxc_deploy_before/btrfs_management.pl
[WARN]        Floday::Lib::Linux::LXC:       End pre deployment hook.
[WARN]        Floday::Lib::Linux::LXC:       Start deploying LXC spyzone-web-galek container.
[WARN]        Floday::Lib::Linux::LXC:        End deploying LXC container.
[WARN]        Floday::Lib::Linux::LXC:       Start post deployment hook.
[WARN]        Floday::Lib::Linux::LXC:       End post deployment hook.
[WARN]                 Floday::Deploy:       Start running setups scripts.
[INFO]                 Floday::Deploy:        Running script: /etc/floday/containers/jaxe/children/core/setups/networking.pl
[INFO]                 Floday::Deploy:        Running script: /etc/floday/containers/jaxe/children/core/setups/cleaning.pl
[INFO]                 Floday::Deploy:        Running script: /etc/floday/containers/jaxe/children/perl/setups/cpan.pl
[INFO]                 Floday::Deploy:        Running script: /etc/floday/containers/jaxe/children/core/setups/updater.pl
[INFO]                 Floday::Deploy:        Running script: /etc/floday/containers/jaxe/children/perl/setups/updater.pl
[INFO]                 Floday::Deploy:        Running script: /etc/floday/containers/jaxe/children/www/children/galek/setups/galek_install.pl
[WARN]                 Floday::Deploy:       End running setups scripts.
[WARN]                 Floday::Deploy:       Start running end_setups scripts.
[WARN]                 Floday::Deploy:       End running end_setups scripts.
[WARN]                 Floday::Deploy:      Start running end_setups scripts.
[WARN]                 Floday::Deploy:      End running end_setups scripts.
[WARN]                 Floday::Deploy:    End deployment of spyzone applications.
[WARN]                 Floday::Deploy:    Start running end_setups scripts.
[INFO]                 Floday::Deploy:     Running script: /etc/floday/containers/jaxe/end_setups/alpine_endpreparation.pl
[INFO]                 Floday::Deploy:     Running script: /etc/floday/containers/jaxe/end_setups/iptables_save.pl
[WARN]                 Floday::Deploy:    End running end_setups scripts.
[WARN]                 Floday::Deploy:   websites deployed.
\end{lstlisting}


Le listing~\ref{fig_1.4_floday} présente une sortie représentative d'un déploiement.
Les messages sont découpée en plusieurs colonnes, dont voici description~:
\begin{itemize}
	\item La première n'est qu'une indiquation quant à la sévérité du log.
	\item La seconde défini le module dans lequel le message est émit.
	\item La dernière présente le message. Notez que l'indentation dans l'alignement permet de voir le degré d'imbrication. Grace à cela, nous pouvons rapidement voir, par exemple, que le script \emph{networking.pl} du log ligne~53 aura été exécuté dans le cadre de l'exécution des scripts de setup sur l'\gls{application} \emph{website-web\_application\_galek} contraint par \emph{website-web\_application} sur l'\gls{hote} \emph{websites}.
\end{itemize}

Pour le moment, cette sortie est également disponible via syslog, bien que le format sera peut-être amené à évolué sur celle-ci.

Attention, notez qu'il y actuellement un bug générant une erreur dans le fichier de log lors du déploiement de l'hôte (présente à la ligne~10 sur le listing~\ref{fig_1.4_floday} celui-ci devrait être résolu un jour%
\footnote{Lien vers le ticket présentant l'anomalie~: \url{https://dev.spyzone.fr/floday/ticket/31}}.
