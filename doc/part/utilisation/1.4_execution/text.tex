\subsection{L'exécution}

\subsubsection{La gestion des logs}
\existstill{1.0.0}

Une fois notre \emph{runfile} et les applications qui s'y réfèrent de définies, il ne reste plus qu'à amorcer le tout et permettre à \emph{Floday} d'effectuer sa principale raison d'être~!
Pour cela nous, n'avons qu'à lancer la commande {\tt floday} avec l'option {\tt --host} permettant de définir quel \gls{hote} du runfile nous voulons déployer.
Des logs d'exécution seront ensuite affichés sur \emph{STDOUT} ainsi que via \emph{syslog}.

\paragraph{Gestion du niveau des logs}
Il n'est malheureusement pas possible actuellement de définir soi-même le niveau de verbosité à avoir sans toucher au code de l'application, mais ce problème sera peut-être résolu un jour%
\footnote{Un ticket a été créé sur le bugtracker pour rendre la gestion du niveau de log plus pratique~: \url{https://dev.spyzone.fr/floday/ticket/29}}.

Actuellement, pour le modifier, il faudra donc changer la valeur de la ligne 18 du fichier \emph{floday.pl}.
Les niveaux disponibles sont ceux habituellement en vigeur sur n'importe quel système \emph{UNIX}.
Pour plus de détails, voir le hash \emph{\%SYSLOG\_PRIORITY\_MAPPER} dans le fichier \emph{Floday/Helper/Logging.pm}.

\begin{lstlisting}[caption={config.yml}, label=fig_1.4_floday, basicstyle=\tiny, xleftmargin=-3cm]
# ./floday.pl --host websites
[WARN]                 Floday::Deploy:   Deploying websites host
[WARN]                 Floday::Deploy:    Start running setups scripts.
[INFO]                 Floday::Deploy:     Running script: /etc/floday/containers/jaxe/setups/iptables_flush.pl
[INFO]                 Floday::Deploy:     Running script: /etc/floday/containers/jaxe/setups/dns.pl
[INFO]                 Floday::Deploy:     Running script: /etc/floday/containers/jaxe/setups/ssh.pl
[INFO]                 Floday::Deploy:     Running script: /etc/floday/containers/jaxe/setups/network.pl
[INFO]                 Floday::Deploy:     Running script: /etc/floday/containers/jaxe/children/perl/setups/cpan.pl
[INFO]                 Floday::Deploy:     Running script: /etc/floday/containers/jaxe/children/perl/setups/updater.pl
 [ERR]        Floday::Lib::Linux::LXC:      lxc-attach: failed to get the init pid
[INFO]                 Floday::Deploy:     Running script: /etc/floday/containers/jaxe/setups/updater.pl
[INFO]                 Floday::Deploy:     Running script: /etc/floday/containers/jaxe/setups/alpine_preparation.pl
[INFO]                 Floday::Deploy:     Running script: /etc/floday/containers/jaxe/setups/debian_preparation.pl
[WARN]                 Floday::Deploy:    End running setups scripts.
[WARN]                 Floday::Deploy:    Start deployment of websites applications.
[WARN]                 Floday::Deploy:     Launching website-web_application application.
[WARN]        Floday::Lib::Linux::LXC:      Start pre destruction hooks.
[INFO]        Floday::Lib::Linux::LXC:       Running script: jaxe/children/core/hooks/lxc_destroy_before/btrfs_management.pl
[WARN]        Floday::Lib::Linux::LXC:      End pre destruction hooks.
[WARN]        Floday::Lib::Linux::LXC:      Start LXC container website-web_application destruction.
[WARN]        Floday::Lib::Linux::LXC:      End LXC container destruction.
[WARN]        Floday::Lib::Linux::LXC:      Start post destroy hook.
[WARN]        Floday::Lib::Linux::LXC:      End post destroy hook.
[WARN]        Floday::Lib::Linux::LXC:      Start pre deployment hooks.
[INFO]        Floday::Lib::Linux::LXC:       Running script: jaxe/children/core/hooks/lxc_deploy_before/btrfs_management.pl
[WARN]        Floday::Lib::Linux::LXC:      End pre deployment hook.
[WARN]        Floday::Lib::Linux::LXC:      Start deploying LXC website-web_application  container.
[WARN]        Floday::Lib::Linux::LXC:       End deploying LXC container.
[WARN]        Floday::Lib::Linux::LXC:      Start post deployment hook.
[WARN]        Floday::Lib::Linux::LXC:      End post deployment hook.
[WARN]                 Floday::Deploy:      Start running setups scripts.
[INFO]                 Floday::Deploy:       Running script: /etc/floday/containers/jaxe/children/core/setups/networking.pl
[INFO]                 Floday::Deploy:       Running script: /etc/floday/containers/jaxe/children/core/setups/cleaning.pl
[INFO]                 Floday::Deploy:       Running script: /etc/floday/containers/jaxe/children/core/setups/updater.pl
[INFO]                 Floday::Deploy:       Running script: /etc/floday/containers/jaxe/children/www/setups/lighttpd.pl
[WARN]                 Floday::Deploy:      End running setups scripts.
[WARN]                 Floday::Deploy:      Launching website-web_application-galek application.
[WARN]        Floday::Lib::Linux::LXC:       Start pre destruction hooks.
[INFO]        Floday::Lib::Linux::LXC:        Running script: jaxe/children/core/hooks/lxc_destroy_before/btrfs_management.pl
[WARN]        Floday::Lib::Linux::LXC:       End pre destruction hooks.
[WARN]        Floday::Lib::Linux::LXC:       Start LXC container spyzone-web-galek destruction.
[WARN]        Floday::Lib::Linux::LXC:       End LXC container destruction.
[WARN]        Floday::Lib::Linux::LXC:       Start post destroy hook.
[WARN]        Floday::Lib::Linux::LXC:       End post destroy hook.
[WARN]        Floday::Lib::Linux::LXC:       Start pre deployment hooks.
[INFO]        Floday::Lib::Linux::LXC:        Running script: jaxe/children/core/hooks/lxc_deploy_before/btrfs_management.pl
[WARN]        Floday::Lib::Linux::LXC:       End pre deployment hook.
[WARN]        Floday::Lib::Linux::LXC:       Start deploying LXC spyzone-web-galek container.
[WARN]        Floday::Lib::Linux::LXC:        End deploying LXC container.
[WARN]        Floday::Lib::Linux::LXC:       Start post deployment hook.
[WARN]        Floday::Lib::Linux::LXC:       End post deployment hook.
[WARN]                 Floday::Deploy:       Start running setups scripts.
[INFO]                 Floday::Deploy:        Running script: /etc/floday/containers/jaxe/children/core/setups/networking.pl
[INFO]                 Floday::Deploy:        Running script: /etc/floday/containers/jaxe/children/core/setups/cleaning.pl
[INFO]                 Floday::Deploy:        Running script: /etc/floday/containers/jaxe/children/perl/setups/cpan.pl
[INFO]                 Floday::Deploy:        Running script: /etc/floday/containers/jaxe/children/core/setups/updater.pl
[INFO]                 Floday::Deploy:        Running script: /etc/floday/containers/jaxe/children/perl/setups/updater.pl
[INFO]                 Floday::Deploy:        Running script: /etc/floday/containers/jaxe/children/www/children/galek/setups/galek_install.pl
[WARN]                 Floday::Deploy:       End running setups scripts.
[WARN]                 Floday::Deploy:       Start running end_setups scripts.
[WARN]                 Floday::Deploy:       End running end_setups scripts.
[WARN]                 Floday::Deploy:      Start running end_setups scripts.
[WARN]                 Floday::Deploy:      End running end_setups scripts.
[WARN]                 Floday::Deploy:    End deployment of spyzone applications.
[WARN]                 Floday::Deploy:    Start running end_setups scripts.
[INFO]                 Floday::Deploy:     Running script: /etc/floday/containers/jaxe/end_setups/alpine_endpreparation.pl
[INFO]                 Floday::Deploy:     Running script: /etc/floday/containers/jaxe/end_setups/iptables_save.pl
[WARN]                 Floday::Deploy:    End running end_setups scripts.
[WARN]                 Floday::Deploy:   websites deployed.
\end{lstlisting}


Le listing~\ref{fig_1.4_floday} présente une sortie représentative d'un déploiement.
Les messages sont découpée en plusieurs colonnes, dont voici description~:
\begin{itemize}
	\item La première n'est qu'une indiquation quant à la sévérité du log.
	\item La seconde défini le module dans lequel le message est émit.
	\item La dernière présente le message. Notez que l'indentation dans l'alignement permet de voir le degré d'imbrication. Grace à cela, nous pouvons rapidement voir, par exemple, que le script \emph{networking.pl} du log ligne~53 aura été exécuté dans le cadre de l'exécution des scripts de setup sur l'\gls{application} \emph{website-web\_application\_galek} contraint par \emph{website-web\_application} sur l'\gls{hote} \emph{websites}.
\end{itemize}

Pour le moment, cette sortie est également disponible via syslog, bien que le format sera peut-être amené à évolué sur celle-ci.

Attention, notez qu'il y actuellement un bug générant une erreur dans le fichier de log lors du déploiement de l'hôte (présente à la ligne~10 sur le listing~\ref{fig_1.4_floday} celui-ci devrait être résolu un jour%
\footnote{Lien vers le ticket présentant l'anomalie~: \url{https://dev.spyzone.fr/floday/ticket/31}}.
