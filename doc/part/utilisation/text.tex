\section{Utilisation de Floday}

\begin{intro}
	Nous verrons dans cette première partie tout ce qui relatif à l'utilisation de \emph{Floday}~:
	\begin{itemize}
		\item Dans un premier temps, les principes de bases seront exposés afin de comprendre avec précision les tenants et aboutissants du logiciel.
		\item Ensuite, un exemple d'implémentation sera présenté.
		\item Enfin, une liste exaustive des fonctionnalités sera évoquée.
	\end{itemize}
\end{intro}

\subsection{Comprentre les principes de base}

\subsubsection{Une mise en situation}
\existstill{1.0.0}

Prenons la problématique suivante~:
Kevin (le nom du personnage n'a pas été choisi par hasard), un passionné d'informatique s'adonne souvant aux tests de beaucoup de logiciels.
La plupart auront été déployés avec une méthode proche de la \emph{RACHE}%
\footnote{La \emph{RACHE} est une méthodologie de génie logiciel assez controversée. Plus d'information sur ce site~: \url{http://www.la-rache.com/}}%
, et ce dans des temps reculés dont toutes traces du mode opératoire auront étés oubliés le jour malheureux ou l'auteur réalisera qu'ils ne fonctionnent plus comme il s'y attendait.

De plus, cette perte de repère est amplifiée par la grosse hétérogénéité des applications~: certaines ont leur configuration dans {\tt/etc}, et leurs données dans {\tt/home/\$USER}, d'autres seront plutôt {\tt/home/\$USER/\emph{<service>}/.config} pour la configuration et {\tt/usr/var/} pour les données, etc.
Sans oublié les scripts au fonctionnement variés qui peuvent être présent n'importe où (des règles \emph{iptables} dans {\tt/etc/init.d/80-custom}, des scripts de backup aux quatres coins de l'arborécences, etc.

Quant au matériel, il est en général du même acabit~: un Rasperry Pi dans une boite en carton%
\footnote{C'est une auto-promotion concernant un de mes anciens serveur de backup~: \url{http://blog.spyzone.fr/2013/11/utiliser-un-raspberry-pi-comme-serveur-de-backup/}}%
, un dédié moisi qui doit tourner dans une cave d'un pays indéterminé ou alors un bout de cyberespace qu'on squatte sans réel garantie qu'on y aura encore accès dans cinq minutes.
Bref\dots{} L'infrastructure physique n'étant pas pérenne, on sera un jour ou l'autre amené à tout redéployer ailleurs ce qui, on s'en doute, ne sera probablement pas réalisé de façon optimale aux vus des remarques précédentes.

Toutes ces difficutés peuvent être acceptables (bien que peu régouissantes) si l'on conserve le postulat de base, celui qui dit que le seul but de cette démarche est de tester.
En réalité, on se retrouve vite à utiliser ces services en production.
Bien-sûr, l'échelle reste en général très limitée quant aux personnes impactés en cas d'éventuelle avarie car elles se limite souvent à l'utilisateur voir son entourage proche, mais elle ne rend pas cette perte négligeable pour autant.
On peut parler ici de serveur emails, serveurs de chats, des forums, des blogs, etc.

Pouvons-nous réellement prendre le risque de perdre tout ces «\,petits\,» services~?
Non bien-sûr (enfin, ça ne serait en général pas la mort non plus de tout perdre ou d'avoir un long temps d'arrêt, mais ça ferait quand-même bien chier)
du coup on rajoute encore plus de complexité au bordel ambiant pour essayer de garantir une certaine résilience~: mise en place d'un système de backup, du monitoring, une gestion avancée de la sécurité (utilisation d'apparmor par exemple), etc.
Puis on fait des incantations aux dieux que l'on vénère pour espérer ne jamais avoir à affronter ce genre de cataclysme car souvent nous n'avons pas nous-même confiance en notre propre infrastructure (il serait prétentieux de parler de stratégie de restauration), tellement celle-ci est foireuse.
C'est triste, n'est-ce-pas~? En tout cas, c'est contre ça que \emph{Floday} essaye de luter.

\subsubsection{À quoi \emph{Floday} répond}
\existstill{1.0.0}
Les principaux objectifs de \emph{Floday} sont les suivants~:
