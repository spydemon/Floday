\section{Utilisation de \emph{Floday}}

\begin{intro}
	Nous verrons dans cette partie tout ce qui est relatif à l'utilisation de \emph{Floday}~:
	\begin{itemize}
		\item Dans un premier temps, les principes de bases seront exposés afin de comprendre avec précision les tenants et aboutissants du logiciel.
		\item Ensuite, les différents éléments de configuration seront présentés, à fin de voir comment rendre le logiciel utilisable avec nos besoins.
		\item Puis un exemple d'implémentation sera introduit pour évoquer de façon exaustive ce que le logiciel permet.
		\item Finalement on verra comment installer le projet.
	\end{itemize}
\end{intro}

\subsection{Les différents éléments}
\existstill{1.0.0}

Nous allons à présent faire le tour des éléments centaux de la solution.

\subsubsection{Le \emph{runfile}}
\existstill{1.0.0}

\begin{lstlisting}[caption={Exemple de runfile.yml}, label=fig_1.3_runfile]
---
hosts:
	websites:
		parameters:
			type:          riuk
			external_ipv4: 192.168.15.151
		applications:
			web_application:
				parameters:
					ipv4:    10.0.3.5
					gateway: 10.0.3.1
					type:    web
				applications:
					my_blog:
						parameters:
							ipv4:     10.0.3.6
							gateway:  10.0.3.1
							type:     wordpress
							data:     /var/www/my_blog
							hostname: blog.spyzone.fr
					mum_blog:
						parameters:
							ipv4:     10.0.3.7
							gateway:  10.0.3.1
							type:     pluxml
							data:     /var/www/mum_blog
							hostname: mum.spyzone.fr
	backup:
		parameters:
			type: jaxe
			external_ipv4: 192.168.15.141
		applications:
			backup_application:
				parameters:
					type: backup_web
\end{lstlisting}


Il s'agit du cœur de l'application~: le fichier qui défini avec précision ce qui doit être déployé.
La figure~\ref{fig_1.3_runfile} est un modèle qu'on utilisera comme exemple.

On y voit la définition de deux \gls{hote}~: \emph{websites} et \emph{backup}.
Dans le premier, on y peut y voir trois \glspl{application}~:\emph{web}, qui est \gls{gestionnaire} de \emph{my\_blog} et de \emph{mum\_blog}.
Cela signifie que \emph{web} peut se baser sur ses \gls{contraint} pour se configurer correctement.
\subsubsection{La définition de conteneurs}

\subsubsection{L'amorçage}

\subsection{Les différents éléments}
\existstill{1.0.0}

Nous allons à présent faire le tour des éléments centaux de la solution.

\subsubsection{Le \emph{runfile}}
\existstill{1.0.0}

\begin{lstlisting}[caption={Exemple de runfile.yml}, label=fig_1.3_runfile]
---
hosts:
	websites:
		parameters:
			type:          riuk
			external_ipv4: 192.168.15.151
		applications:
			web_application:
				parameters:
					ipv4:    10.0.3.5
					gateway: 10.0.3.1
					type:    web
				applications:
					my_blog:
						parameters:
							ipv4:     10.0.3.6
							gateway:  10.0.3.1
							type:     wordpress
							data:     /var/www/my_blog
							hostname: blog.spyzone.fr
					mum_blog:
						parameters:
							ipv4:     10.0.3.7
							gateway:  10.0.3.1
							type:     pluxml
							data:     /var/www/mum_blog
							hostname: mum.spyzone.fr
	backup:
		parameters:
			type: jaxe
			external_ipv4: 192.168.15.141
		applications:
			backup_application:
				parameters:
					type: backup_web
\end{lstlisting}


Il s'agit du cœur de l'application~: le fichier qui défini avec précision ce qui doit être déployé.
La figure~\ref{fig_1.3_runfile} est un modèle qu'on utilisera comme exemple.

On y voit la définition de deux \gls{hote}~: \emph{websites} et \emph{backup}.
Dans le premier, on y peut y voir trois \glspl{application}~:\emph{web}, qui est \gls{gestionnaire} de \emph{my\_blog} et de \emph{mum\_blog}.
Cela signifie que \emph{web} peut se baser sur ses \gls{contraint} pour se configurer correctement.
\subsubsection{La définition de conteneurs}

\subsubsection{L'amorçage}

\subsection{Les différents éléments}
\existstill{1.0.0}

Nous allons à présent faire le tour des éléments centaux de la solution.

\subsubsection{Le \emph{runfile}}
\existstill{1.0.0}

\begin{lstlisting}[caption={Exemple de runfile.yml}, label=fig_1.3_runfile]
---
hosts:
	websites:
		parameters:
			type:          riuk
			external_ipv4: 192.168.15.151
		applications:
			web_application:
				parameters:
					ipv4:    10.0.3.5
					gateway: 10.0.3.1
					type:    web
				applications:
					my_blog:
						parameters:
							ipv4:     10.0.3.6
							gateway:  10.0.3.1
							type:     wordpress
							data:     /var/www/my_blog
							hostname: blog.spyzone.fr
					mum_blog:
						parameters:
							ipv4:     10.0.3.7
							gateway:  10.0.3.1
							type:     pluxml
							data:     /var/www/mum_blog
							hostname: mum.spyzone.fr
	backup:
		parameters:
			type: jaxe
			external_ipv4: 192.168.15.141
		applications:
			backup_application:
				parameters:
					type: backup_web
\end{lstlisting}


Il s'agit du cœur de l'application~: le fichier qui défini avec précision ce qui doit être déployé.
La figure~\ref{fig_1.3_runfile} est un modèle qu'on utilisera comme exemple.

On y voit la définition de deux \gls{hote}~: \emph{websites} et \emph{backup}.
Dans le premier, on y peut y voir trois \glspl{application}~:\emph{web}, qui est \gls{gestionnaire} de \emph{my\_blog} et de \emph{mum\_blog}.
Cela signifie que \emph{web} peut se baser sur ses \gls{contraint} pour se configurer correctement.
\subsubsection{La définition de conteneurs}

\subsubsection{L'amorçage}

\subsection{Les différents éléments}
\existstill{1.0.0}

Nous allons à présent faire le tour des éléments centaux de la solution.

\subsubsection{Le \emph{runfile}}
\existstill{1.0.0}

\begin{lstlisting}[caption={Exemple de runfile.yml}, label=fig_1.3_runfile]
---
hosts:
	websites:
		parameters:
			type:          riuk
			external_ipv4: 192.168.15.151
		applications:
			web_application:
				parameters:
					ipv4:    10.0.3.5
					gateway: 10.0.3.1
					type:    web
				applications:
					my_blog:
						parameters:
							ipv4:     10.0.3.6
							gateway:  10.0.3.1
							type:     wordpress
							data:     /var/www/my_blog
							hostname: blog.spyzone.fr
					mum_blog:
						parameters:
							ipv4:     10.0.3.7
							gateway:  10.0.3.1
							type:     pluxml
							data:     /var/www/mum_blog
							hostname: mum.spyzone.fr
	backup:
		parameters:
			type: jaxe
			external_ipv4: 192.168.15.141
		applications:
			backup_application:
				parameters:
					type: backup_web
\end{lstlisting}


Il s'agit du cœur de l'application~: le fichier qui défini avec précision ce qui doit être déployé.
La figure~\ref{fig_1.3_runfile} est un modèle qu'on utilisera comme exemple.

On y voit la définition de deux \gls{hote}~: \emph{websites} et \emph{backup}.
Dans le premier, on y peut y voir trois \glspl{application}~:\emph{web}, qui est \gls{gestionnaire} de \emph{my\_blog} et de \emph{mum\_blog}.
Cela signifie que \emph{web} peut se baser sur ses \gls{contraint} pour se configurer correctement.
\subsubsection{La définition de conteneurs}

\subsubsection{L'amorçage}

\subsection{Les différents éléments}
\existstill{1.0.0}

Nous allons à présent faire le tour des éléments centaux de la solution.

\subsubsection{Le \emph{runfile}}
\existstill{1.0.0}

\begin{lstlisting}[caption={Exemple de runfile.yml}, label=fig_1.3_runfile]
---
hosts:
	websites:
		parameters:
			type:          riuk
			external_ipv4: 192.168.15.151
		applications:
			web_application:
				parameters:
					ipv4:    10.0.3.5
					gateway: 10.0.3.1
					type:    web
				applications:
					my_blog:
						parameters:
							ipv4:     10.0.3.6
							gateway:  10.0.3.1
							type:     wordpress
							data:     /var/www/my_blog
							hostname: blog.spyzone.fr
					mum_blog:
						parameters:
							ipv4:     10.0.3.7
							gateway:  10.0.3.1
							type:     pluxml
							data:     /var/www/mum_blog
							hostname: mum.spyzone.fr
	backup:
		parameters:
			type: jaxe
			external_ipv4: 192.168.15.141
		applications:
			backup_application:
				parameters:
					type: backup_web
\end{lstlisting}


Il s'agit du cœur de l'application~: le fichier qui défini avec précision ce qui doit être déployé.
La figure~\ref{fig_1.3_runfile} est un modèle qu'on utilisera comme exemple.

On y voit la définition de deux \gls{hote}~: \emph{websites} et \emph{backup}.
Dans le premier, on y peut y voir trois \glspl{application}~:\emph{web}, qui est \gls{gestionnaire} de \emph{my\_blog} et de \emph{mum\_blog}.
Cela signifie que \emph{web} peut se baser sur ses \gls{contraint} pour se configurer correctement.
\subsubsection{La définition de conteneurs}

\subsubsection{L'amorçage}

\subsection{Les différents éléments}
\existstill{1.0.0}

Nous allons à présent faire le tour des éléments centaux de la solution.

\subsubsection{Le \emph{runfile}}
\existstill{1.0.0}

\begin{lstlisting}[caption={Exemple de runfile.yml}, label=fig_1.3_runfile]
---
hosts:
	websites:
		parameters:
			type:          riuk
			external_ipv4: 192.168.15.151
		applications:
			web_application:
				parameters:
					ipv4:    10.0.3.5
					gateway: 10.0.3.1
					type:    web
				applications:
					my_blog:
						parameters:
							ipv4:     10.0.3.6
							gateway:  10.0.3.1
							type:     wordpress
							data:     /var/www/my_blog
							hostname: blog.spyzone.fr
					mum_blog:
						parameters:
							ipv4:     10.0.3.7
							gateway:  10.0.3.1
							type:     pluxml
							data:     /var/www/mum_blog
							hostname: mum.spyzone.fr
	backup:
		parameters:
			type: jaxe
			external_ipv4: 192.168.15.141
		applications:
			backup_application:
				parameters:
					type: backup_web
\end{lstlisting}


Il s'agit du cœur de l'application~: le fichier qui défini avec précision ce qui doit être déployé.
La figure~\ref{fig_1.3_runfile} est un modèle qu'on utilisera comme exemple.

On y voit la définition de deux \gls{hote}~: \emph{websites} et \emph{backup}.
Dans le premier, on y peut y voir trois \glspl{application}~:\emph{web}, qui est \gls{gestionnaire} de \emph{my\_blog} et de \emph{mum\_blog}.
Cela signifie que \emph{web} peut se baser sur ses \gls{contraint} pour se configurer correctement.
\subsubsection{La définition de conteneurs}

\subsubsection{L'amorçage}

\subsection{Les différents éléments}
\existstill{1.0.0}

Nous allons à présent faire le tour des éléments centaux de la solution.

\subsubsection{Le \emph{runfile}}
\existstill{1.0.0}

\begin{lstlisting}[caption={Exemple de runfile.yml}, label=fig_1.3_runfile]
---
hosts:
	websites:
		parameters:
			type:          riuk
			external_ipv4: 192.168.15.151
		applications:
			web_application:
				parameters:
					ipv4:    10.0.3.5
					gateway: 10.0.3.1
					type:    web
				applications:
					my_blog:
						parameters:
							ipv4:     10.0.3.6
							gateway:  10.0.3.1
							type:     wordpress
							data:     /var/www/my_blog
							hostname: blog.spyzone.fr
					mum_blog:
						parameters:
							ipv4:     10.0.3.7
							gateway:  10.0.3.1
							type:     pluxml
							data:     /var/www/mum_blog
							hostname: mum.spyzone.fr
	backup:
		parameters:
			type: jaxe
			external_ipv4: 192.168.15.141
		applications:
			backup_application:
				parameters:
					type: backup_web
\end{lstlisting}


Il s'agit du cœur de l'application~: le fichier qui défini avec précision ce qui doit être déployé.
La figure~\ref{fig_1.3_runfile} est un modèle qu'on utilisera comme exemple.

On y voit la définition de deux \gls{hote}~: \emph{websites} et \emph{backup}.
Dans le premier, on y peut y voir trois \glspl{application}~:\emph{web}, qui est \gls{gestionnaire} de \emph{my\_blog} et de \emph{mum\_blog}.
Cela signifie que \emph{web} peut se baser sur ses \gls{contraint} pour se configurer correctement.
\subsubsection{La définition de conteneurs}

\subsubsection{L'amorçage}

