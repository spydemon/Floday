\subsection{Installation de \emph{Floday}}

\subsubsection{Quelle version choisir~?}

\emph{Floday} utilise le système de normalisation des numéros de version sémantique%
\footnote{En savoir plus sur le versionnage sémentique~: \url{http://semver.org/}}.
Cela signifie que la version du logiciel sera toujours données sous la forme de trois entiers~: {\tt{}x.y.z}.
Voici leur significations~:

\begin{description}
	\item[x] correspond à la version majeur. Deux versions majeurs différentes ne sont pas interopérables.
Il faudra donc s'attendre à ce qu'il y ait de la casse lors d'une mise à jour  de version majeures.
	\item[y] correspond à la version mineure, soit l'ajout de nouvelles fonctionnalités mais qui ne (devraient) pas casser le code existant. Celle-ci devrait donc vous apporter de nouvelles fonctionnalités, ou en tout cas devrait changer la façon dont une partie du logiciel procède.
	\item[z] correspond aux correctifs. Il est recommandé de toujours effectuer les mises à jour de correctifs car ils ne sont utilisés que pour corriger un fonctionnment sensé déjà être fonctionnel, les bogues quoi.
\end{description}

Avec ces informations en tête, libre à vous de chosir le cycle d'évolution que vous préférez. Une brance existe pour chaque cycle. Nous avons donc actuellement les branches~:
\begin{description}
	\item[master] qui correspondra toujours à la dernière version publiée du logiciel, en prenant en compte les évolution de versions majeures.
	\item[v1] qui correspondra toujours à la version mineure la plus élevée, mais de la version majeure 1. Votre configuration ne devrait donc pas se voir incompatibles après ce genre de mise à jour.
	\item[v1.0] qui ne prendra en compte que les correctifs de sécurité ou de bogues.
\end{description}

Quand une version 1.1 de \emph{Floday} sera \gls{propulse}, nous aurons de la même sorte, une nouvelle branche \emph{v1.1} de publiée, et il en sera de même pour les autres.
Notez que vous pouvez aussi directement récupérer une version en question (par exemple {\tt git checkout 1.0.1}) car un tag sera à chaque fois associé.

En tant qu'utilisateur, vous serez probablement intéressé par l'évolution de la version majeure en cours d'utilisation. Si vous choisissez de mettre à jour directement le logiciel via le dépôt Git, vous pouvez puller régulièrement la branche portant le nom de la version (par exemple v1).
Si par contre vous maintenez ce logiciel pour le compte d'une distribution quelconque, vous pouvez continuer à suivre les corrections d'anomalies après le freeze des évolutions en vérifiant régulièrement les nouveaux commits de la branche ayant le nom de la version majeur présente sur la distribution (par exemple v1.1).

\subsubsection{Comment l'installer~?}

Pour le moment, je ne vous cache pas que c'est bien la merde.
Mais là aussi, des efforts seront peut-être (probablement) fait un jour.

\paragraph{Dépendances nécessaires pour \emph{Floday}}
\begin{itemize}
	\item Perl~5 (version 5.20 minimum).
	\item \texttt{apt-get install -y --no-install-recommends bridge-utils cgroup-tools\\cgroupfs-mount curl apparmor apparmor-utils lxc}
\end{itemize}

\paragraph{Processus d'installation}

\begin{itemize}
	\item Il est conseillé de cloner le dépôt de \emph{Floday}, par exemple dans \path{/opt} et de le déployer sur le commit correspondant à la version du logiciel que vous désirez utiliser.
	\item À présent, il faut exécuter le fichier \path{install.pl} présent à la racine du dépôt.
		Celui-ci «\,installera\,» les modules propres de \emph{Floday} dans un endroit accessible par Perl (\path{/etc/perl}) et téléchargera tous les modules via le \emph{CPAN} nécessaires.
	\item Faire un lien symbolique entre \path{/opt/floday/lxc-template/lxc-flodayalpine} et \path{/usr/lib/lxc/template/lxc-flodayalpine} pour qu'il soit directement accesible depuis \path{LXC}.
\end{itemize}

À présent, tout devrait être bon. Pour être sûr, vous pouvez configurer votre machine comme une interface de développement (voir le paragraphe en question) et jouer les tests d'intégration.
Mais c'est probablement un peu lourd et foireux de faire ça sur votre futur machine de production\dots
