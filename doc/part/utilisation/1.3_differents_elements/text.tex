\subsection{Les éléments centraux}

\subsubsection{Le fichier de configuration}
\existstill{1.0.0}
Un petit fichier de configuration existe à l'emplacement {\tt/etc/floday/config.cfg}.
Celui-ci est au format \emph{INI}%
\footnote{Courte définition du format \emph{INI} sur le site de \emph{Wikipedia}~:
	\url{https://fr.wikipedia.org/wiki/Fichier_INI}
}.
Il s'agit du seul fichier dont l'emplacement soit imposé.
Voici un descriptif de ce qu'il contient~:
\newline

\begin{tabular}{|l|l|p{4cm}|p{4.1cm}|}
	\hline
	Séction & Paramètre & Valeur par défaut & Description \\
	\hline
	containers & path & /etc/floday/containers & Emplacement du jeu de conteneurs.\\
	floday & runfile & /etc/floday/runfile.yml & Emplacement du \gls{runfile}. \\
	logging & metadata\_folder & /tmp/floday/logging & Dossier utilisé de façon interne par \emph{Floday} pour gérer l'indentation des messages de logs. \\
	\hline
\end{tabular}
\newline

Pour le moment, il est aussi possible d'avoir des valeurs de configuration utilisées par le \gls{jeu_conteneur}, voir par l'\gls{initialisation}.
Je ne suis pas certain qu'il s'agisse de la meilleur façon de faire, du coup il faut s'attendre à ce que cela change peut-être un jour.

Voici les paramètres de configuration existants actuellement pour  \emph{flodayalpine}, l'initialisation livré par défaut avec \emph{floday}~:
\newline

\begin{tabular}{|l|l|p{4cm}|p{5.25cm}|}
	\hline
	Séction & Paramètre & Valeur par défaut & Description \\
	\hline
	LXC & cache\_folder & /tmp/floday/lxc-flodayalpine & Dossier de cache utilisé pour stocker les logiciels déjà téléchargés.\\
	LXC & id\_groups & 1000 & Nombre de groupes d'uid et de gid disponibles pour l'\gls{application} en cours d'\gls{instantiation}. Un groupe est choisi au hasard parmi ceux qui ne sont pas encore utilisés.\\
	LXC & id\_range & 100000 & Nombre de uid et de gid disponibles au sein d'un même groupe.\\
	LXC & repo & dl-4.alpinelinux.org & Dépôt sur lequel récupérer les packages à installer.\\
	\hline
\end{tabular}
\newline

\subsubsection{Le \emph{runfile}}
\existstill{1.0.0}

\begin{figure}
	\centerfloat
	\caption{Illustrations des relations inter-conteneurs et des notions de chemin d'applications et de conteneurs.}
	\label{relation_conteneurs}
	\usetikzlibrary{decorations}
\usetikzlibrary{decorations.text}

\begin{tikzpicture}
	%%%%%%%%%%%
	% Runfile %
	%%%%%%%%%%%
	\draw[color=gray!70] (-.75, 2.75) rectangle (5.25, -2);
	\node[text width=4.25cm, right, text centered] at (0, 2.5) {Runfile};
	\node[text width=4.25cm, right, text justified] at (0, 0) {%
		\begin{lstlisting}[basicstyle=\scriptsize]
host:
  spyzone:
    parameters:
      type: riuk
    applications:
      web:
        parameters:
          type: lighttpd
        applications:
          my_blog:
            parameters:
              type: wordpress
		\end{lstlisting}
	};

	%%%%%%%%%%%%%%%%%%%%
	% riuk-core config %
	%%%%%%%%%%%%%%%%%%%%
	\draw[color=gray!70] (9.5, 4.5) rectangle (15, 2);
	\node[text width=4cm, right, text centered] at (10.25, 4) {riuk-core config.yml};
	\node[text width=4cm, right, text justified] at (10.25, 3) {%
		\begin{lstlisting}[basicstyle=\scriptsize]
setup:
  ssh:
    exec: /.../ssh.pl
    priority: 1-10
[...]
		\end{lstlisting}
	};

	%%%%%%%%%%%%%%%%%%%%%%%%
	% riuk-lighttpd config %
	%%%%%%%%%%%%%%%%%%%%%%%%
	\draw[color=gray!70] (13.5, 0.45) rectangle (18.5, -2.75);
	\node[text width=4cm, right, text centered] at (14, 0) {riuk-lighttpd config.yml};
	\node[text width=4cm, right, text justified] at (14, -1.5) {%
		\begin{lstlisting}[basicstyle=\scriptsize]
inherit:
  - riuk-core
setup:
  lighttpd:
    exec: /.../lighty.pl
    priority: 2-10
[...]
		\end{lstlisting}
	};

	%%%%%%%%%%%%%%%%%%%%%%%%%%%%%%%%%%
	% riuk-lighttpd-wordpress config %
	%%%%%%%%%%%%%%%%%%%%%%%%%%%%%%%%%%
	\draw[color=gray!70] (6, 0.45) rectangle (10.85, -2.75);
	\node[text width=4cm, right, text centered] at (6.5, 0) {riuk-lighttpd-wordpress config.yml};
	\node[text width=4cm, right, text justified] at (6.5, -1.5) {%
		\begin{lstlisting}[basicstyle=\scriptsize]
inherit:
  - riuk-core
setup:
  wordpress:
    exec: /.../wp.pl
    priority: 2-10
[...]
		\end{lstlisting}
	};

	%%%%%%%%%%%%%%%%%%%%%%%%%%%
	% Relations de soumission %
	%%%%%%%%%%%%%%%%%%%%%%%%%%%
	\draw [->, >=latex, color=gray!80] (2.80, 0.65) to[out=0, in=360] (4.2, -0.4);
	\draw [->, >=latex, color=gray!80] (4.2, -0.5)
	  to[out=0, in=180] (6, 1)
	  to (12, 1)
	  to[out=0] (14, 0);
	\draw[color=gray!80] (8, 1.25) node {Chemin de conteneur};
	\draw [->, >=latex, color=gray!80] (4.2, -0.55) to[out=0, in=360] (5, -1.55);
	\draw [->, >=latex, color=gray!80] (5, -1.65) to[out=0, in=180] (7, 0);

	%%%%%%%%%%%%%%%%%%%%%%%
	% Relation d'héritage %
	%%%%%%%%%%%%%%%%%%%%%%%
	\draw [->, >=latex, color=gray!80] (9, -1)
	  -- (11.5, -1)
	  .. controls +(.75, 0) and +(0, -.75)
	  .. (12.25, 0)
	  -- (12.25, 2);
	\draw [->, >=latex, color=gray!80] (14.2, -1)
	  -- (13, -1)
	  .. controls +(-.75, 0) and +(0, -.75)
	  .. (12.25, 0)
	  -- (12.25, 2);
	  %to[out=-90] (12.25, 2);
	  %to[out=0, in=-90] (12.25, 2);
	\draw [color=gray!80] (13.25, 1.5) node {Hérite de};

	%%%%%%%%%%%%%%%%
	% Applications %
	%%%%%%%%%%%%%%%%
	\draw (8.425, -4) node[draw=gray!80] (srw) {spyzone-web-wordpress};
	\draw (16, -4) node[draw=gray!80] (sr) {spyzone-web};
	\draw [->, >=latex, color=gray!80] (8.425, -2.75) -- (8.425, -3.75);
	\draw [->, >=latex, color=gray!80] (16, -2.75) -- (16, -3.75);
	\draw (12.2125, -3.25) node {Initialisation des applications};

	%%%%%%%%%%%%%%%%%%%%%%%%%%
	% Chemins d'applications %
	%%%%%%%%%%%%%%%%%%%%%%%%%%
	\draw [->, >=latex, color=gray!80] (0.45, 1.2) to[out=-180, in=-180] (1, 0.15);
	\draw [->, >=latex, color=gray!80] (1, 0.05) to[out=-180, in=-180] (1.8, -0.95);
	\draw[->, >=latex, color=gray!80,%
	  postaction={decorate, decoration={%
		  raise=1pt, text along path, text align=center, text={|\color{gray!80}|Chemin d'application ||}
	  }}
	 ] (1.8, -1.05) to[out=-180, in=-180] (srw);
	\draw [->, >=latex, color=gray!80, rounded corners=5pt] (1, 0.1)
	  -| (0, -4.75)
	  -| (sr);
	\draw[color=gray!80] (2, -4.5) node {Chemin d'application};

	%%%%%%%%%%%%%%%
	% Contraintes %
	%%%%%%%%%%%%%%%
	\draw [->, >=latex, color=gray!80, dashed,
	  postaction={decorate, decoration={%
		  raise=1pt, text along path, text align=center, text={|\color{gray!80}|Est contraint par ||}
	  }}
	  ] (srw.5) -- (sr.172.30);
	\draw [<-, >=latex, color=gray!80, dashed,
	  postaction={decorate, decoration={%
		  raise=-8pt, text along path, text align=center, text={|\color{gray!80}|Est gestionnaire de  ||}
	  }}
	  ] (srw.-5) -- (sr.-172.30);
\end{tikzpicture}

\end{figure}
\begin{lstlisting}[caption={runfile.yml}, label=fig_1.3_runfile]
---
hosts:
	websites:
		parameters:
			type:          riuk
			external_ipv4: 192.168.15.151
		applications:
			web_application:
				parameters:
					ipv4:    10.0.3.5
					gateway: 10.0.3.1
					type:    web
				applications:
					my_blog:
						parameters:
							ipv4:     10.0.3.6
							gateway:  10.0.3.1
							type:     wordpress
							data:     /var/www/my_blog
							hostname: blog.spyzone.fr
					mum_blog:
						parameters:
							ipv4:     10.0.3.7
							gateway:  10.0.3.1
							type:     pluxml
							data:     /var/www/mum_blog
							hostname: mum.spyzone.fr
	backup:
		parameters:
			type: jaxe
			external_ipv4: 192.168.15.141
		applications:
			backup_application:
				parameters:
					type: backup_web
\end{lstlisting}


Il s'agit du cœur de l'application~: le fichier qui défini avec précision ce qui doit être déployé.
Le code~\ref{fig_1.3_runfile} est un modèle qu'on utilisera comme exemple.

On constate tout d'abord que ce fichier est en \emph{YAML}.
C'est le cas pour l'ensemble des fichiers de configuration présent dans \emph{Floday}.
On y voit ensuite la définition de deux \glspl{hote}~: \emph{websites} et \emph{backup}.
Dans le premier, trois \glspl{application} y sont définies~:\emph{web}, qui est \gls{gestionnaire} des deux autres~: \emph{my\_blog} et \emph{mum\_blog}.
Cela signifie que \emph{web} peut se baser sur ses \glspl{contraint} pour se configurer correctement.
La figure~\ref{relation_conteneurs} essaye de représenter visuellement ces notions.

L'\gls{instantiation} elle aussi est hiérarchique~: dans un premier temps, l'hôte sera déployé, puis viendra le tour de \emph{web\_application} et enfin celui de \emph{my\_blog} et de \emph{num\_blog}.
Notez d'ailleurs que l'ordre de ces deux derniers est aléatoire.
Ce n'est donc pas parce que \emph{my\_blog} a été défini avant qu'il sera forcément déployé en premier.

Nous pouvons voir que pour chaque hôte ou application deux clefs principales y sont définies~:
\begin{description}
	\item[applications] défini toutes les applications qui seront contraintes par celle en cours de \gls{definition}.
	\item[parameters] surcharge la valeur de \glspl{param_applicatifs} définis dans le conteneur.
		Le seul paramètre applicatif obligatoire est \emph{name} car il permet d'identifier le nom de l'application et de former son \gls{chemin_application}.
		Tous les autres dépendent du conteneur en cours de définition.
\end{description}

\subsubsection{La définition de conteneurs}
\existstill{1.0.0}

Dans le paragraphe précédent, on a vu comment définir les \glspl{application} que l'on voudrait voir déployer sur un hôte bien défini.
Il nous reste à présent à définir en quoi elles consistent concrètement.

Pour cela, la première notion à introduire et celle de \gls{chemin_conteneur}.
Comme le glossaire l'indique, il ne s'agit ni plus, ni moins qu'une agrégation des types de l'\gls{imbrication} courante.
Ce chemin permet de trouver facilement le fichier {\tt config.yml} du conteneur en question.

En prennant à nouveau l'exemple du code~\ref{fig_1.3_runfile}, on y défini un conteneurs de chemin \emph{riuk-web-wordpress}, un autre de chemin \emph{riuk-web-pluxml}, le gestionnaire \emph{riuk-web}, et enfin, un hôte \emph{riuk}.
Avec le \gls{jeu_conteneur} à l'emplacement par défaut, on peut trouver la \gls{definition} de notre premier conteneur dans le fichier \path{/etc/floday/containers/jaxe/children/web/children/wordpress/config.yml}.
Le code~\ref{fig_1.3_config} sera utilisé pour illustrer ce qui est attendu de ce fichier.

\begin{lstlisting}[caption={config.yml}, label=fig_1.3_config]
inherit:
	-jaxe-core
setups:
	deploy:
		exec:     riuk/children/web/children/wordpress/setups/deploy.pl
		priority: 10
	import:
		exec:     riuk/children/web/children/wordpress/setups/import.pl
		priority: 30
parameters:
	public:
		mandatory: false
		pattern: ^(true|false)$
	hostname:
		mandatory: true
		pattern:   ^[\w.-]$
	template:
		mandatory: true
		value:     flodayalpine -- version 3.4
hooks:
	lxc_deploy_before:
		open_firewall:
			exec:     riuk/children/[~\dots~]/wordpress/hooks/lxc_deploy_before/of.pl
			priority: 10
	lxc_deploy_after:
		close_firewall:
			exec:     riuk/children/[~\dots~]/wordpress/hooks/lxc_deploy_after/cf.pl
			priority: 10
	lxc_destroy_before:
		clear_filesystem:
			exec:     riuk/children/[~\dots~]/wordpress/hooks/lxc_destroy_before/cf.pl
			priority: 10
	lxc_destroy_after:
		update_fstab:
			exec:     riuk/children/[~\dots~]/wordpress/hooks/lxc_destroy_after/uf.pl
			priority: 10
\end{lstlisting}


\paragraph{Le nœud \emph{hooks}}
Il s'agit ici de pouvoir greffer des scripts durant le processus d'\emph{initialisation}.
Ils ne doivent être utilisés que dans les cas ou des actions doivent êtres prises durant cette phase.
Les actions devraient rester le plus possible cantonées au nœud \emph{setups} et \emph{end\_setups} car il devient vite complexe de comprendre correctement l'étape de déploiement d'un conteneur si beaucoup de hooks dépendant du contexte y sont greffés.
Notez qu'ils sont aussi implémenté au niveau de la méthode d'\gls{initialisation}, et donc dépendent de celle-ci.

Les attributs des hooks sont les mêmes que ceux des nœuds \emph{setups} et \emph{end\_setups}.
Voici la liste des hooks existant pour l'initialisation via le template \emph{LXC} \emph{flodayalpine}, utilisé par défaut~:

\begin{description}
	\item[lxc\_deploy\_before] Les scripts seront exécutés avant {\tt lxc-deploy}.
	\item[lxc\_deploy\_after] Les scripts seront exécutés après {\tt lxc-deploy}.
	\item[lxc\_destroy\_before] Les scripts seront exécutés avant {\tt lxc-destroy}.
	\item[lxc\_destroy\_after] Les scripts seront exécutés après {\tt lxc-destroy}.
\end{description}

\paragraph{Le nœud \emph{inherit}}
Cette liste comprend les \glspl{chemin_conteneur} de l'ensemble des parents du conteneur courant.
Attention, l'ordre dans lesquels apparaissent les parents est ignoré.
Il est donc impossible de le prévoir, ce qui posera des problèmes si les deux parents définissent un même \gls{attribut}  qui n'est pas réécrit dans le conteneur courant.

\paragraph{Le nœud \emph{parameters}}
Nous y définissons les différents paramètres accessibles dans les scripts d'installation. Leurs valeurs peuvent être surchargées au niveau du \gls{runfile}.
Chaque \gls{param_applicatifs} peut avoir ces attributs~:
\begin{description}
	\item[mandatory] Peut valoir \emph{true} ou \emph{false}%
		\footnote{Attention, le support du type booléan en \emph{YAML} étant assez hasardeux en Perl~5, nous utilisons dans \emph{Floday} des chaines de carractères valant soit «\,true\,» soit «\,false\,». On ne peut donc pas utiliser les autres formes officielles acceptées en \emph{YAML} pour des booléen.}%
		. Si l'attribut est obligatoire mais non défini lors du déploiment, une erreur sera émise par \emph{Floday} et tout le processus sera annulé.
	\item[pattern] Cet attribut peut contenir une expression rationnelle \emph{PCRE} à laquelle sera soumit l'attribut. Si le test échoue, le déploiment sera là aussi annulé.
	\item[value] Défini une valeur par défaut à l'attribut. Si celui-ci ne se retrouve pas surchargé au niveau du \gls{runfile}, c'est cette valeur qui sera utilisée.
\end{description}

\paragraph{Le nœud \emph{setups}}
Une fois l'\gls{initialisation} complète, les scripts présents dans cette partie sont executés les uns après les autres pour finir le \gls{deploiement}.
Chaque script peut avoir ces attributs~:
\begin{description}
	\item[exec] Chemin du script a exécuter. Celui-ci doit être exécutable par l'utilisateur effectuant le déploiment. Il peut être présent n'importe où du moment qu'il soit accessible. Une convention veut cependant qu'il soit présent dans le dossier \emph{setups} au même emplacement que l'est le fichier \emph{config.yml} en cours d'écriture.
	\item[priority] Permet de définir la priorité d'exécution du script pour les cas ou l'ordre ait une importance. Attention à bien prendre en compte les priorités des scripts définis dans les éventuels conteneurs parents. Les scripts sont exécutés par ordre croissant.
\end{description}

\paragraph{Le nœud \emph{end\_setup}}
Tout comme \emph{setup}, celui-ci possède des informations sur les scripts à exécuter lors du déploiment.
La différence vient du fait qu'ici, ils sont exécutés une fois l'ensemble des applications contraintes déployées et non avant.
Les attributs sont les mêmes que pour le nœud \emph{setup} et \emph{hooks}.

\subsubsection{Écrire les scripts d'exécution}
Il s'agit ici de présenter un exemple de script pouvant être présent dans un nœud \emph{setup}, \emph{end\_setup} ou \emph{hook} dans la \gls{definition} d'un \gls{conteneur}.
Le code~\ref{fig_1.3_setup} est un exemple de script permettant de configurer \emph{lighttpd} de façon à ce qu'il puisse communiquer avec l'extérieur et qu'il puisse faire le proxy entre les différentes applications qu'il \gls{contraint}.

\begin{lstlisting}[caption={Exemple de script de setup}, label=fig_1.3_setup]
#!/usr/bin/env perl

use strict;
use warnings;
use v5.20;

use Backticks;
use Floday::Setup;

$Backticks::autodie = 1;

#########################
# Lighttpd installation #
#########################
my $lxc = $APP->getLxcInstance();
$lxc->start() if $lxc->is_stopped();
$lxc->exec('apk add lighttpd');
$lxc->exec('rc-update add lighttpd');
$lxc->exec('/etc/init.d/lighttpd start');

#####################################
# Lighttpd configuration management #
#####################################
$APP->generateFile(
  'jaxe/children/www/setups/lighttpd/lighttpd.conf',
  undef,
  '/etc/lighttpd/lighttpd.conf'
);
for ($APP->getApplications()) {
	$APP->generateFile(
	  $_->getParameter('lighttpd_config'),
	  {$_->getParameters()},
	  '/etc/lighttpd/conf.d/'.$_->getInstancePath().'.conf'
	);
}

#################################
# Lighttpd routing instructions #
#################################
my $ipv4 = $APP->getParameter('networking_ipv4');
my ($ipv6) = $APP->getParameter('networking_ipv6') =~$\sim$~ /^(.*)\//;
`iptables -t nat -A PREROUTING ! -i lxcbr0 -p tcp --dport 80 -j DNAT --to-dest $ipv4`;
`iptables -t filter -A FORWARD ! -i lxcbr0 -p tcp --dport 80 -j ACCEPT`;
`ip6tables -t nat -A PREROUTING ! -i lxcbr0 -p tcp --dport 80 -j DNAT --to-dest $ipv6`;
`ip6tables -t filter -A FORWARD ! -i lxcbr0 -p tcp --dport 80 -j ACCEPT`;
\end{lstlisting}


Tout d'abord, on constate que ce script est écrit en \emph{Perl~5}.
Ce n'est absolument pas une obligation, la seule contrainte étant que ce fichier ait les droits d'exécutions par l'utilsateur exécutant \emph{Floday}.
Par contre, li s'agit du seul langage a posséder le module \emph{Floday::Setup}, bien pratique pour travailler sur l'\gls{application} en cours d'instantiation.
Pour une documentation exaustive concernant ce que le module vous offre, je vous invite à lire sa page de documentation~:
\path{perldoc Floday::Setup}.

La première partie intéressante débute ligne~15, en montrant comment faire pour manipuler l'application depuis le script.
On voit que l'on passe par la variable {\tt\$APP} automatiquement déclarée par \emph{Floday::Setup}.
Dès la ligne~16, on l'utilise pour rapatrirer un objet \emph{Linux::LXC} qui nous permet de toucher directement le conteneur \emph{LXC} qui aura à charge notre application, ce qui est illustré les trois lignes suivantes.

La ligne~24 présente une autre fonctionnalité du module \emph{Floday::Setup}, celle de facilitier la gestion des fichier de configuration.
Mais là encore, il est conseillé de se référer à \emph{perldoc} pour avoir des détails quant à l'utilisation qui devrait en être faite.

La ligne~29 montre comment nous devons procéder pour agir non pas sur l'application en cours de déploiement, mais sur une autre (ici, ceux qu'il gère).

TODO: parler aussi du debug :
%root@spyzone:/home/spydemon# /etc/floday/containers/jaxe/children/backup_client/end_setup/volume_mounter.pl --container spyzone-backup

