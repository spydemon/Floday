\subsection{Écrire son jeu de conteneurs}
\existstill{1.0.0}

\emph{Floday} ne prévoit pas de distribuer en plus de cette mécanique, des conteneurs concrets que vous pourriez utiliser directement.
C'est à vous d'écrire les vôtres, de façon à ce qu'ils conviennent parfaitement à votre architecture, et que vous sachiez précisément ce qu'ils font~!
Vous pouvez tout de même vous inspirer du jeu de conteneurs \path{riuk}, présent dans le dossier \path{floday/containers/riuk} pour avoir des indications sur la façon de procéder.

Globalement, le seul fichier imposé est le \emph{config.yml} définissant le conteneur.
Les scripts d'exécutions peuvent être localisés où vous le souhaitez, bien que conserver une structure cohérente (comme celle présente dans \emph{riuk}) peut être conseillé.
De la même façon, les scripts peuvent être écrits dans le langage que vous voulez, la seule contrainte est que ceux-ci aient les droits d'exécutions par l'utilisateur exécutant \emph{Floday}.

Utiliser Perl pour les scripts peut néanmoins être un choix à privilégier, car le module \path{Floday::Setup} permet de facilement accéder à l'\gls{application} en cours de déploiement, à ses \glspl{param_applicatifs} ainsi qu'à quelques autres outils comme la génération de fichiers.
La documentation du module étant complète, je vous invite à faire un \path{perldoc Floday::Setup} pour avoir une définition exhaustive de ce qu'il propose.

Par défaut, le \gls{jeu_conteneur} devra être présent dans le dossier \path{/etc/floday/containers}.
Une fois vos conteneurs d'écrits, il ne vous restera plus qu'à faire le \gls{runfile} qui se chargera de les transformer en applications~!
