\subsection{Écrire son jeu de conteneurs}
\existstill{1.0.0}

\emph{Floday} ne prévoit pas de distribuer en plus de cette mécanique, des conteneurs concrêt que vous pouriez utiliser directement.
C'est à vous d'écrire les vôtres, de façon à ce qu'ils conviennent parfaitement à votre architecture, et que vous sachiez précisément ce qu'ils font~!
Vous pouvez tout de même vous inspirer du jeu de contenur \path{riuk}, présent dans le dossier \path{floday/containers/riuk} pour savoir comment procéder.

Globalement, le seul fichier imposé est le fichier de configuration définissant le conteneur.
Les scripts d'exécutions ou les paramètres peuvent être défini comme vous le souhaitez.
De la même façon les scripts peuvent être écrits dans le langage que vous voulez, la seul contrainte est que celui-ci ait les droits d'exécutions.

Utiliser Perl pour les scripts peut néamoins être un choix à privilégier car le module \path{Floday::Setup} permet de façillement accéder à l'\gls{application} en cours de déploiement, à ses \gls{param_applicatifs} ainsi qu'à quelques autres outils.
La documentation du module étant complète, je vous invite à faire un \path{perldoc Floday::Setup} pour avoir une définition exaustive de ce qu'il propose.

Par défaut, le \gls{jeu_conteneur} devra être présent dans le dossier \path{/etc/floday/containers}.
Une fois vos conteneurs d'écrits, il ne vous restera plus qu'à faire le \gls{runfile} qui se chargera de les transformer en applications~!
