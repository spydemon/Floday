\subsection{Comprendre les principes de base}
\subsubsection{Une mise en situation}
\existstill{1.0.0}

Prenons la problématique suivante~:
Kevin (le nom du personnage n'a pas été choisi par hasard), un passionné d'informatique s'adonne souvent aux tests de beaucoup de logiciels.
La plupart auront été déployés avec une méthode proche de la \emph{RACHE}%
\footnote{La \emph{RACHE} est une méthodologie de génie logiciel assez controversée. Plus d'information sur ce site~: \url{http://www.la-rache.com/}}%
, et ce dans des temps reculés dont toutes traces du mode opératoire auront étés oubliées le jour malheureux ou l'auteur réalisera qu'ils ne fonctionnent plus comme il s'y attendait.

De plus, cette perte de repères est amplifiée par la grosse hétérogénéité des applications~: certaines ont leurs fichiers de configuration dans le répertoire {\tt/etc} et leurs données dans {\tt/home/\$USER}, d'autres seront plutôt {\tt/home/\$USER/\emph{<service>}/.config} pour la configuration et {\tt/var/lib/} pour les données, etc.
Sans oublier les scripts aux fonctionnements variés qui peuvent être présents n'importe où (des règles \emph{iptables} dans {\tt/etc/init.d/80-custom}, des programmes de backup aux quatre coins de l'arborescence, etc.).

Quant au matériel, il est en général du même acabit~: un Raspberry Pi dans une boite en carton%
\footnote{C'est une autopromotion concernant un de mes anciens serveurs de backup~: \url{http://blog.spyzone.fr/2013/11/utiliser-un-raspberry-pi-comme-serveur-de-backup/}}%
, un dédié moisi qui doit tourner dans une cave d'un pays indéterminé ou alors un bout de cyberespace qu'on squatte sans réelle garantie qu'on y aura encore accès dans cinq minutes.
Bref\dots{} L'infrastructure physique n'étant pas pérenne, on sera un jour ou l'autre amené à tout redéployer ailleurs ce qui, on s'en doute, ne sera probablement pas effectué de façon optimale à la vue des remarques précédentes.

Toutes ces difficultés peuvent être acceptables (bien que peu réjouissantes) si l'on conserve le postulat de base, celui qui dit que le seul but de cette démarche est de tester.
En réalité, on se retrouve vite à employer ces instances en production.
Bien sûr, l'échelle reste en général très petite quant aux personnes impactées en cas d'éventuelle avarie, car elles se limitent souvent à l'utilisateur voir son entourage proche, mais elle ne rend pas cette perte négligeable pour autant.
On peut parler ici de serveurs d'emails ou de clavardage, des forums, des blogs, etc.

Pouvons-nous réellement prendre le risque de perdre tout ces «\,petits\,» services~?
Non. Certes, ça ne serait en général pas non plus la mort, mais ça ferait tout de même bien chier.
Du coup, on rajoute encore plus de complexité au désordre ambiant pour essayer de garantir une certaine résilience~: mise en place d'un système de backup, du monitoring, une gestion avancée de la sécurité (utilisation d'\emph{AppArmor} par exemple), etc.
Puis on fait des incantations aux dieux que l'on vénère pour espérer ne jamais avoir à affronter ce genre d'accident, car souvent nous n'avons pas nous-mêmes confiance en notre propre infrastructure (il serait prétentieux de parler de stratégie de restauration), tellement celle-ci est bancale.
C'est triste, n'est-ce pas~? En tout cas, c'est contre ça que \emph{Floday} essaye de lutter.

\subsubsection{À quoi \emph{Floday} répond}
\existstill{1.0.0}
De façon un peu moins romancée que lors du précédent paragraphe, on peut résumer \emph{Floday} aux points suivants~:

\paragraph{Regrouper la configuration logicielle}
Le système a été pensé pour contenir toute la configuration et tous les scripts d'installation et de fonctionnement qu'un logiciel requière, cantonné à un même endroit.
De cette façon, indépendamment de la complexité de celui-ci, la zone de recherche dans laquelle investiguer ou agir lors d'évolutions sur le service en question reste très délimitée et à l'écart du bruit introduit par l'extérieur.
Il s'agit ici de la notion de \glspl{conteneur}.

\paragraph{Clarifier par confinement}
Chaque \gls{application} doit pouvoir fonctionner indépendamment des autres, dans une portée bien définie.
La communication interapplication n'est prévue pour se faire que via sockets fichiers ou réseaux explicites.
C'est pour cela que la conteneurisation via \emph{LXC} semble être un bon choix.

\paragraph{Uniformiser l'infrastructure}
Un mécanisme d'héritage est également présent au niveau des \glspl{conteneur}, permettant de factoriser certaines tâches et du coup, de favoriser l'uniformité.
Ainsi, nous pouvons imaginer le fait que plusieurs conteneurs aient besoin du réseau, ou d'un accès \emph{ssh}.
Ces tâches pourront donc être héritées d'un conteneur parent rendant leur fonctionnement identique, ou au moins similaire.

\paragraph{Faciliter la configuration}
Trouver le juste milieu entre une uniformité excessive et un désordre entropique est souvent délicat.
C'est avec cette remarque en contexte que le système de configuration, complexe au premier abord, mais semblant répondre aux nécessités d'adaptations, a été pensé.
En effet, chaque \gls{conteneur} est défini en fonction de nombreux \glspl{attribut}.
Tous peuvent se voir réécrits dans des \glspl{sous-conteneur}.
Les \glspl{param_applicatifs} peuvent même l'être au niveau du \gls{runfile}.

\paragraph{Clarifier l'interaction logicielle}
Chaque \gls{conteneur} peut être \gls{gestionnaire} et \gls{sous-conteneur}.
L'interaction entre eux est forte, c'est-à-dire qu'un sous-conteneur n'est pas prévu pour pouvoir fonctionner sans son gestionnaire.
Il est également possible d'avoir une contrainte sur plusieurs niveaux~: c'est-à-dire, de permettre au sous-conteneur d'être lui aussi gestionnaire.
Nous pouvons prendre l'exemple d'un conteneur \emph{Wordpress}, chargé de déployer le \emph{CMS} éponyme, dont le gestionnaire serait \emph{Web}, un conteneur ayant rôle de proxy \emph{HTTP}.
Cette hiérarchisation est explicite depuis le \gls{runfile} et depuis le \gls{jeu_conteneur}.

\paragraph{Faciliter la gestion multihôte}
Un seul \gls{runfile} devrait être utilisé par infrastructure, même si celle-ci s'appuie sur plusieurs hôtes réels.
\emph{Floday} permet de définir celui que l'on s'apprête à déployer.
Encore une fois, le fait d'avoir toute la configuration dans un seul fichier facilite la prise de conscience de l'intégralité des acteurs.
Imaginons le cas d'un serveur de backup~: il aura directement accès aux autres \gls{application} à backuper.

\paragraph{Permettre une bonne résilience}
Le déploiement avec \emph{Floday} n'est pas universel, dans le sens ou il pourrait se faire partout, indépendamment du socle technologique ou physique sur lequel nous nous appuyons.
Par contre, il se veut entièrement automatisé, sous réserve que le nouvel hôte respecte des postulats de base propres à chaque \gls{jeu_conteneur}.
De cette façon, si une panne physique survient sur une machine, les services pourront facilement être redéployés ailleurs et remis en route.
