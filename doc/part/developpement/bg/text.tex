\subsection{Le bug tracker}
\label{sec:contribution_bt}
\existstill{1.0.0}

Le bug tracker est disponible à l'adresse~: \url{https://dev.spyzone.fr/floday/query}.
C'est l'endroit privilégié d'échange entre utilisateurs et développeurs de l'application.
En effet, on y trouve toutes les demandes d'évolutions, les constats d'anomalies, ou des questions dont la réponse n'est pas encore dans ce guide.
Ces tickets peuvent donc être de trois types~:

\subsubsection{Les types de tickets}
\existstill{1.0.0}

\paragraph{Les tickets d'évolution} Ils représentent les nouvelles fonctionnalités à implémenter dans le programme. Ces développements seront à merger dans la prochaine version majeure ou mineure publiée, et doivent généralement être documentés dans ce manuel.
\paragraph{Les tickets d'anomalies} Ils représentent les bogues repérés dans le logiciel. Quand un correctif est disponible, il est livré sur chaque branche en ayant besoin.
\paragraph{Les tickets de documentation} Il représentent un manquement quant à la compréhension de quelque chose. Ils peuvent être utilisés pour poser une question sur un fonctionnement resté flou.
Ces tickets doivent mener à une amélioration de ce document ainsi que des pages de manuels des différents modules Perl mis à disposition via \emph{Floday}.

\subsubsection{La rédaction d'un ticket}
\existstill{1.0.0}

Attention, il est préférable que vous écrivez en anglais lors de la création de votre ticket.
Voici une description des informations qui vous seront demandées~:
\newline

\begin{longtable}{|p{3cm}|p{10cm}|}
	\hline
	Champ à renseigner & Description\\
	\hline
	Résumé & Sujet du ticket, soit une phrase qui décrit en une ligne de quoi il parle.\\
	Rapporteur & Laisser la valeur par défaut (Spydemon). Il n'y a que moi qui travaille sur le projet pour le moment de toute façon.\\
	Description & Il faut expliquer avec le plus détails possibles l'objet de votre demande ici afin que les autres (moi~?) aient le plus de chances de la comprendre.\\
	Type & Les différents types disponibles sont listés au paragraphe précédent.\\
	Priorité & Sauf ticket d'anomalie critique, la priorité doit toujours être à \emph{normale}.\\
	Version & La version que vous êtes en train d'utiliser. À noter que pour les tickets d'évolutions, il vaut mieux que celle-ci soit la plus récente possible.\\
	Composant & Défini à quelle partie du logiciel le ticket fait référence. Pour le moment, aucune segmentation n'est présente au niveau du projet, il faudra donc toujours sélectionner \emph{core}.\\
	Copie à & Permet de mettre d'autres destinataires en copie des emails envoyés lors de l'évolution du ticket.\\
	Propriétaire & Il s'agit de vous, vous pouvez laisser la valeur \emph{<default>}.\\
	\hline
\end{longtable}

\subsubsection{Le workflow du bug tracker}
\existstill{1.0.0}

Une fois votre ticket créé, il aura le statut \emph{ouvert}.
Il restera dans cet état jusqu'à ce qu'il soit compris et prit en compte par l'équipe concernée (moi~?).
Avant cette étape, des échanges auront peut-être lieu dans les commentaires pour essayer d'en savoir plus.
Il pourra à ce moment passer soit à l'état \emph{accepté}, ce qui signifiera qu'il sera pris en compte un jour, soit \emph{rejeté}.
Dans le second cas, la demande sera ignorée à tout jamais.
Si le ticket a été accepté, il prendra l'état \emph{en cours de traitement} lorsque quelqu'un commencera à travailler dessus, avant de finir en \emph{terminé} une fois le développement de publié.
À ce moment, il sera ajouté à un jalon définissant à partir de quelle version le contenu relatif au ticket sera intégré à \emph{Floday}.
