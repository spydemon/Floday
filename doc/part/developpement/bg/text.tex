\subsection{Le bug tracker}

Le bug tracker est disponible à l'adresse~: \url{https://dev.spyzone.fr/floday/query}.
C'est l'endroit privilégié d'échange entre utilisateurs et développeurs de l'application.
En effet, on y trouve toutes les demandes d'évolutions, les constats d'anomalies, ou encore des questions dont la réponse n'a pas été trouvé dans ce guide.
Ces tickets peuvent donc être de trois types~:

\subsubsection{Les types de tickets}
\paragraph{Les tickets d'évolution} Ils représentent les nouvelles fonctionnalités à implémenter dans le programme. Ces développements seront à merger sur la prochaine version majeure publiée, et doivent généralement être documentés dans ce document
\paragraph{Les tickets d'anomalies} Ils représentent les anomalies repérés dans le logiciel. Une fois l'anomalie détectées, celle-ci est corrigé à l'aide d'un correctif écrit pour l'occasion, et publié sur chaque version en ayant besoin.
\paragraph{Les ticktes de documentation} Il présentent un manquement quant à la compréhention de quelque chose. Ces tickets peuvent être utilisés pour poser une question sur le fonctionnement d'une partie du logiciel non comprise et dont l'information n'a été trouvée nul part. Ces tickets doivent mener à une amélioration de ce document ainsi des pages de manuels des différents modules Perl mis à disposition via \emph{Floday}.

\subsubsection{La rédaction d'un ticket}
Attention, il est préférable que vous écrivez en anglais lors de la création de votre ticket.
Voici une description des informations qui vous seront demandées~:
\newline

\begin{tabular}{|p{3cm}|p{10cm}|}
	\hline
	Champ à renseigner & Description\\
	\hline
	Résumé & Sujet du ticket, soit une phrase qui décrit en une ligne de quoi il parle.\\
	Rapporteur & Laisser la valeur par défaut (Spydemon). Il n'y a que moi qui travaille sur le projet pour le moment de toute façon.\\
	Description & Il faut expliquer avec le plus détails possibles l'objet de votre demande ici afin que les autres (moi~?) aient le plus de chances possible de la comprendre.\\
	Type & Les différents types disponibles sont expliqués au paragraphe précédent.\\
	Priorité & Sauf ticket d'anomalie critique, la priorité doit toujours être à \emph{normale}.\\
	Version & La version que vous êtes en train d'utiliser. À noter que pour les tickets d'évolutions, il vaut mieux que celle-ci soit la plus récente possible.\\
	Composant & Défini sur quelle partie du logiciel le ticket fait référence. Pour le moment, aucune segmentation n'est présente au niveau du projet, il faudra donc toujours sélectionner \emph{core}.\\
	Copie à & Permet de mettre d'autre destinataires en copie des emails envoyés lors de l'évolution du ticket.\\
	Propriétaire & Il s'agit de vous, vous pouvez donc laisser la valeur \emph{<default>}.\\
	\hline
\end{tabular}
\newline

\subsubsection{Le workflow}

Une fois votre ticket créé, il prendra le status \emph{ouvert}.
Il restera dans cet état jusqu'à ce qu'il soit comprit et pris en compte par l'équipe concernée (moi~?).
Avant ce moment, des échanges auront peut-être lieux dans les commentaires du ticket pour essayer d'en savoir plus.
Il pourra à ce moment passer soit à l'état \emph{accepté}, ce qui signifiera qu'il sera pris en compte un jour, soit \emph{rejeté}.
Dans ce cas, le ticket sera ignoré pour la suite des dévelopements.
Si le ticket a été accepté, quand sa résolution sera en cours, il prendra d'état \emph{en cours de traitement}, avant de finir en \emph{terminé}.
À ce moment, il sera intégré à un jalon définissant à partir de quelle version le développement relatif au ticket sera intégrer à \emph{Floday}.

\subsection{Démarrer une instance de dévelopment}

La première étape pour contribuer et de réussir à exécuter \emph{Floday} dans un environnement contrôlé sur lequel des tests peuvent être exécutés sans craintes.
Une façon de faire va être décrite dans cette section.

Pour fonctionner, on utilisera deux images \emph{VirtualBox} dont l'une sera un clone le la première.
Cette façon de procéder nous permet d'avoir un «\,effet mémoire\,» permettant d'exécuter \emph{Floday} sur une machine virtuelle et de pouvoir encore après-coup s'y connecter pour vérifier avec précision ce qui a été effectué.
Si un test veut être refait, il nous suffira de détruire cette VM et d'en recloner une nouvelle basée sur celle de base.
La machine virtuelle de base s'appellera \emph{Floday\_Clean} et la copie de travail courante \emph{Floday\_Work}.
Toute la configuration sera à faire sur \emph{Floday\_Clean} car cette machine servant de base à \emph{Floday\_Work}, tout son contenu y sera déployé.
À l'inverse, les modifications apportée à \emph{Floday\_Work} seront perdues dès le prochain clonage.

Il est conseillé d'utiliser Debian 8.0 (Jessie) avec OpenRC comme système d'init.
La création de cette infra reste assez bancale pour le moment, mais elle a le mérite de fonctionner très bien une fois les galères de l'installation passées.
Le jeu en vaut donc la chandelle !
Voici les étapes d'installation~:
\begin{itemize}
	\item Installation de \emph{VirtualBox}.
	\item Création d'un dossier partagé appelé \emph{floday} entre \path{/opt/floday/} sur la VM et le dépôt Git sur l'hôte. Attention, ce partage doit être en lecture et écriture pour pouvoir passer tous les tests unitaires.
	\item Création d'un lien symbolique entre \path{/usr/share/perl5/Virt} et \path{/opt/perlvirtlxc/lib/Virt}.
	\item Configurer les réseaux pour avoir un pond sur \emph{eth0} et un réseau privé hôte sur \emph{eth1}.
	\item Virer \emph{systemd} pour utiliser \emph{Open-RC} à la place~: \url{http://linuxmafia.com/kb/Debian/openrc-conversion.html}.
	\item Installation des additions invités : \url{htps://virtualboxes.org/doc/installing-guest-additions-on-debian/}
	\item Création du dossier \path{/opt/floday}, \path{/var/lib/floday} et \path{/etc/floday}.
	\item Variables d'environnement à ajouter dans \path{/root/.bash}~:\\
{\tt export FLODAY\_CONTAINERS="/opt/floday/src/containers/";\\
export FLODAY\_T="/opt/floday/t/";\\
export FLODAY\_T\_SRC="/opt/floday/src/";}
	\item {\tt{}apt-get install -y --no-install-recommends bridge-utils cgroup-tools\\
		cgroupfs-mount curl apparmor apparmor-utils lxc}
	\item Édition du fichier \emph{/etc/network/interfaces}~:\\
		{\tt{}source /etc/network/interfaces.d/*\\
\\
\# The loopback network interface\\
auto lo\\
iface lo inet loopback\\
\\
\# The primary network interface\\
allow-hotplug eth0\\
\\
iface eth0 inet dhcp\\
iface eth1 inet dhcp\\
\# This is an autoconfigured IPv6 interface\\
iface eth0 inet6 auto\\
\\
auto lxcbr0\\
iface lxcbr0 inet static\\
	address 10.0.3.1\\
	netmask 255.255.255.0\\
		}
	\item Écrire le fichier \path{/etc.init.d/floday}.
	Il faudra penser à changer \emph{192.168.1.12} par l'ip de votre VM appartenant à l'interface bridgé avec votre routeur allant vers Internet.\\
	{\tt \#!/sbin/openrc-run\\
\\
depend() \{\\
	~~need cgroupfs-mount\\
	~~need vboxadd\\
\}\\
\\
start() \{\\
	ebegin "Init Floday stuff"\\
	brctl addbr lxcbr0\\
	ifup lxcbr0\\
	dhclient eth1\\
	mount -t vboxsf floday /opt/floday\\
	mount -t vboxsf perlvirtlxc /opt/perlvirtlxc\\
	echo 1 > /proc/sys/net/ipv4/ip\_forward\\
	iptables -t nat -A POSTROUTING -s 10.0.3.0/24 -o eth0 $\backslash$\\
	-j SNAT --to-source 192.168.1.12\\
	eend 0\\
	\}
}
	\item \path{rc-update add floday}
	\item {\tt ln -s /opt/floday/lxc-templates/lxc-flodayalpine $\backslash$\\ /usr/share/lxc/templates/}
	\item {\tt ln -s /opt/floday/floday.cfg /etc/floday/floday.cfg}
	\item {\tt ln -s /opt/floday/samples/run.xml /etc/floday/run.xml}
	\item Écrire le fichier \path{/etc/lxc/default.conf}~:
		{\tt{}lxc.network.type = veth\\
lxc.network.link = lxcbr0\\
lxc.network.flags = up\\
lxc.network.hwaddr = 00:16:3e:xx:xx:xx}
	\item {\tt cpan Log::Any::Adapter Test::Exception Backticks Moo Config::Tiny File::Slurp YAML::Tiny Template::Alloy Hash::Merge IPC::Run Unix::Syslog}
	\item Ajouter les paramètres \path{apparmor=1 security=apparmor} aux variables \path{GRUB_CMDLINE_LINUX} du fichier \emph{/etc/default/grub}.
	\item {\tt sudo update grub}.
	\item Ajouter la ligne \path{10.0.3.5 test.keh.keh test2.keh.keh} dans le fichier \emph{/etc/hosts} pour pouvoir réussir les tests d'intégration.
	\item Rebooter le conteneur.
\end{itemize}

Pour valider que la configuration soit correcte, il vous est conseillé de jouer les tests d'intégrations.
Si ils ne passent pas malgré le fait que vous ayez scrupuleusement suivit ce guide, vous êtes invités à poster un ticket de type \emph{question} afin que nous puissission enrichir la procédure.

À terme, un script de déploiement automatique sera peut-être fourni.

\subsection{Apporter des modifications au code source}

Une fois que vous accédez au code source de l'application, vous pouvez bien-entendu en faire ce que vous voulez.
Si participer à l'évolution vous intéresse, vous êtes encouragé à me faire des pull requests que je mergerais dans le dépôt principal si votre travail répond correctement à un ticket \emph{accepté}.

N'ayant jamais eu à effectuer ce genre d'actions pour le moment, je ne suis pas sûr qu'il s'agisse de la meilleure façon de faire, ni même qu'elle soit réellement pratique sur le long terme, mais on aura toujours l'occasion de changer ça une ce logiciel massivement maintenu \Winkey.
\subsection{Participer à la documentaton}
