\section{Comment contribuer~?}
	\begin{intro}
	Utiliser une application, c'est bien. Mais contribuer à son amélioration, c'est mieux~!
	Vous pouvez apporter votre pierre à l'édifice via plusieurs façons.
	La première consiste simplement à expliquer via le bug-tracker les difficultés que vous rencontrez, les bogues découvert ou encore les évolutions que vous aimeriez voir.
	La seconde serait de directement apporter vous-même ces modifications au code source du projet.
	Et enfin, une dernière possibilité serait de contribuer à la rédaction de cette documentation, en français, certes, mais aussi dans d'autres langues~!

	Cette section évoquera ces parties via les axes suivant~:
	\begin{itemize}
		\item Comment utiliser le bug-tracker~?
		\item Démarrer une instance de dévelopement de \emph{Floday}.
		\item Contribuer au code source.
		\item Contribuer à la documentation.
	\end{itemize}
\end{intro}

\subsection{Les différents éléments}
\existstill{1.0.0}

Nous allons à présent faire le tour des éléments centaux de la solution.

\subsubsection{Le \emph{runfile}}
\existstill{1.0.0}

\begin{lstlisting}[caption={Exemple de runfile.yml}, label=fig_1.3_runfile]
---
hosts:
	websites:
		parameters:
			type:          riuk
			external_ipv4: 192.168.15.151
		applications:
			web_application:
				parameters:
					ipv4:    10.0.3.5
					gateway: 10.0.3.1
					type:    web
				applications:
					my_blog:
						parameters:
							ipv4:     10.0.3.6
							gateway:  10.0.3.1
							type:     wordpress
							data:     /var/www/my_blog
							hostname: blog.spyzone.fr
					mum_blog:
						parameters:
							ipv4:     10.0.3.7
							gateway:  10.0.3.1
							type:     pluxml
							data:     /var/www/mum_blog
							hostname: mum.spyzone.fr
	backup:
		parameters:
			type: jaxe
			external_ipv4: 192.168.15.141
		applications:
			backup_application:
				parameters:
					type: backup_web
\end{lstlisting}


Il s'agit du cœur de l'application~: le fichier qui défini avec précision ce qui doit être déployé.
La figure~\ref{fig_1.3_runfile} est un modèle qu'on utilisera comme exemple.

On y voit la définition de deux \gls{hote}~: \emph{websites} et \emph{backup}.
Dans le premier, on y peut y voir trois \glspl{application}~:\emph{web}, qui est \gls{gestionnaire} de \emph{my\_blog} et de \emph{mum\_blog}.
Cela signifie que \emph{web} peut se baser sur ses \gls{contraint} pour se configurer correctement.
\subsubsection{La définition de conteneurs}

\subsubsection{L'amorçage}

\subsection{Les différents éléments}
\existstill{1.0.0}

Nous allons à présent faire le tour des éléments centaux de la solution.

\subsubsection{Le \emph{runfile}}
\existstill{1.0.0}

\begin{lstlisting}[caption={Exemple de runfile.yml}, label=fig_1.3_runfile]
---
hosts:
	websites:
		parameters:
			type:          riuk
			external_ipv4: 192.168.15.151
		applications:
			web_application:
				parameters:
					ipv4:    10.0.3.5
					gateway: 10.0.3.1
					type:    web
				applications:
					my_blog:
						parameters:
							ipv4:     10.0.3.6
							gateway:  10.0.3.1
							type:     wordpress
							data:     /var/www/my_blog
							hostname: blog.spyzone.fr
					mum_blog:
						parameters:
							ipv4:     10.0.3.7
							gateway:  10.0.3.1
							type:     pluxml
							data:     /var/www/mum_blog
							hostname: mum.spyzone.fr
	backup:
		parameters:
			type: jaxe
			external_ipv4: 192.168.15.141
		applications:
			backup_application:
				parameters:
					type: backup_web
\end{lstlisting}


Il s'agit du cœur de l'application~: le fichier qui défini avec précision ce qui doit être déployé.
La figure~\ref{fig_1.3_runfile} est un modèle qu'on utilisera comme exemple.

On y voit la définition de deux \gls{hote}~: \emph{websites} et \emph{backup}.
Dans le premier, on y peut y voir trois \glspl{application}~:\emph{web}, qui est \gls{gestionnaire} de \emph{my\_blog} et de \emph{mum\_blog}.
Cela signifie que \emph{web} peut se baser sur ses \gls{contraint} pour se configurer correctement.
\subsubsection{La définition de conteneurs}

\subsubsection{L'amorçage}

\subsection{Les différents éléments}
\existstill{1.0.0}

Nous allons à présent faire le tour des éléments centaux de la solution.

\subsubsection{Le \emph{runfile}}
\existstill{1.0.0}

\begin{lstlisting}[caption={Exemple de runfile.yml}, label=fig_1.3_runfile]
---
hosts:
	websites:
		parameters:
			type:          riuk
			external_ipv4: 192.168.15.151
		applications:
			web_application:
				parameters:
					ipv4:    10.0.3.5
					gateway: 10.0.3.1
					type:    web
				applications:
					my_blog:
						parameters:
							ipv4:     10.0.3.6
							gateway:  10.0.3.1
							type:     wordpress
							data:     /var/www/my_blog
							hostname: blog.spyzone.fr
					mum_blog:
						parameters:
							ipv4:     10.0.3.7
							gateway:  10.0.3.1
							type:     pluxml
							data:     /var/www/mum_blog
							hostname: mum.spyzone.fr
	backup:
		parameters:
			type: jaxe
			external_ipv4: 192.168.15.141
		applications:
			backup_application:
				parameters:
					type: backup_web
\end{lstlisting}


Il s'agit du cœur de l'application~: le fichier qui défini avec précision ce qui doit être déployé.
La figure~\ref{fig_1.3_runfile} est un modèle qu'on utilisera comme exemple.

On y voit la définition de deux \gls{hote}~: \emph{websites} et \emph{backup}.
Dans le premier, on y peut y voir trois \glspl{application}~:\emph{web}, qui est \gls{gestionnaire} de \emph{my\_blog} et de \emph{mum\_blog}.
Cela signifie que \emph{web} peut se baser sur ses \gls{contraint} pour se configurer correctement.
\subsubsection{La définition de conteneurs}

\subsubsection{L'amorçage}

\subsection{Les différents éléments}
\existstill{1.0.0}

Nous allons à présent faire le tour des éléments centaux de la solution.

\subsubsection{Le \emph{runfile}}
\existstill{1.0.0}

\begin{lstlisting}[caption={Exemple de runfile.yml}, label=fig_1.3_runfile]
---
hosts:
	websites:
		parameters:
			type:          riuk
			external_ipv4: 192.168.15.151
		applications:
			web_application:
				parameters:
					ipv4:    10.0.3.5
					gateway: 10.0.3.1
					type:    web
				applications:
					my_blog:
						parameters:
							ipv4:     10.0.3.6
							gateway:  10.0.3.1
							type:     wordpress
							data:     /var/www/my_blog
							hostname: blog.spyzone.fr
					mum_blog:
						parameters:
							ipv4:     10.0.3.7
							gateway:  10.0.3.1
							type:     pluxml
							data:     /var/www/mum_blog
							hostname: mum.spyzone.fr
	backup:
		parameters:
			type: jaxe
			external_ipv4: 192.168.15.141
		applications:
			backup_application:
				parameters:
					type: backup_web
\end{lstlisting}


Il s'agit du cœur de l'application~: le fichier qui défini avec précision ce qui doit être déployé.
La figure~\ref{fig_1.3_runfile} est un modèle qu'on utilisera comme exemple.

On y voit la définition de deux \gls{hote}~: \emph{websites} et \emph{backup}.
Dans le premier, on y peut y voir trois \glspl{application}~:\emph{web}, qui est \gls{gestionnaire} de \emph{my\_blog} et de \emph{mum\_blog}.
Cela signifie que \emph{web} peut se baser sur ses \gls{contraint} pour se configurer correctement.
\subsubsection{La définition de conteneurs}

\subsubsection{L'amorçage}

