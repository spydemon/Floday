\section{Ressources}
\begin{intro}
Avant toute chose, il est de bon ton de prendre un petit moment pour lister les différents éléments qui gravitent au tour de \emph{Floday}.
Ils peuvent se montrer appréciables pour prolonger la compréhension que vous voulez avoir de l'outil, ou pour participer de façon plus active au projet.
\end{intro}

\subsection{Ce document}
\existstill{1.0.0}
Vous êtes en train de lire la documentation du logiciel \emph{Floday}.
Il peut vous sembler étrange que celle-ci soit délivrée sous la forme d'un fichier \emph{PDF}, plutôt que d'être présenté sous une forme plus interactive permise par les technologies du Web.
Ce choix me parait néanmoins pertinent, car il impose la création d'une ressource clairement délimitée censée contenir l'intégralité des informations existantes.
Dans la même philosophie, une version de cette documentation existera pour chaque version de \emph{Floday} délivrée. Il n'y a donc pas de risque à ce que celle-ci commence à diverger avec le logiciel que vous êtes effectivement en train d'utiliser.
Enfin, il me semble plus pratique de n'avoir qu'un seul document à conserver plutôt que de devoir aller cliquer partout le jour ou une information viendra à nous manquer.
Notez aussi que par manque de temps, cette documentation n'existe aujourd'hui qu'en français.

\subsection{La documentation du code}
\existstill{1.0.0}
Le code source du projet en lui-même possède une documentation utile aux développeurs.
Celle-ci est accessible soit en le lisant directement, ou alors à travers l'outil \emph{perldoc}, car \emph{Floday} est intégralement écrit en \emph{Perl 5}.

\subsection{Le bug tracker}
\existstill{1.0.0}
Un bug tracker est disponible à l'adresse \url{https://dev.spyzone.fr/floday/query}.
Comme son nom l'indique, il liste l'intégralité des demandes d'évolutions et des rapports de bogues constatés sur le projet.
C'est aussi un point d'entrée privilégié si cette présente documentation ne répond pas correctement à l'une de vos questions.
En effet, ce document est considéré comme faisant partie intégrante du projet.

Avant de poster un nouveau ticket, il vous est néanmoins demandé de vérifier si aucun doublon ne sera introduit par votre action.
 Limiter le bruit et les redondances est un élément simple et qui permet de faire gagner du temps précieux aux contributeurs qui l'allouerons à meilleur escient ailleurs \smiley.

Ce document présente \hyperref[sec:contribution_bt]{une partie entière} réservée à l'utilisation du bug tracker.
La lire vous est également vivement conseillé si vous comptez participer.

\subsection{Le dépôt \emph{Git}}
\existstill{1.0.0}
Certes, le projet est censé être libre, mais le dépôt \emph{Git} n'est pour l'instant pas totalement public\dots{}
Les raisons derrière cette limitation sont en fait techniques~: actuellement, \emph{Gitolite} est utilisé pour faire la gestion des permissions sur le dépôt.
Or, ce dernier nécessite une connexion via \emph{SSH}, qui, elle-même impose au serveur de posséder votre clef publique.
Sans cette clef, il est donc impossible de récupérer la moindre source.
Bien sûr, je vous donnerais les accès en lecture si vous m'envoyez la vôtre par email.
Ce problème devrait être résolu un jour, car le projet devrait migrer sous \emph{Gitlab}.
