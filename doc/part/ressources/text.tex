\section{Ressources}
\begin{intro}
Avant tout de chose, il est de bon ton de prendre un petit moment consacré à la présentation des différents outils ainsi qu'à l'utilisation pour lesquels ils ont étés pensés.
Il semble aussi relativement pertinant que nous expliquins comment faire pour récupérer \emph{Floday} et le faire fonctionner sur vos ordinateurs.
\end{intro}

\subsection{Ce document}
\existstill{1.0.0}
Vous êtes en train de lire la documentation du logiciel \emph{Floday}.
Il peut vous sembler étrange que celle-ci soit délivrée sous la forme d'un document texte, plutôt que de la présenter sous une forme plus interactive permise par les technologies du Web via la création d'un site dédié, ou au moins d'un wiki.
Ce choix me parait néanmoins pertinant car il permet la création d'une resource clairement délimitée sensée contenir l'intégralité des ressources existantes.
Dans la même philosophie, une version du document existera pour chaque version de \emph{Floday}, il n'y a donc pas de risque à ce que la documentation dont vous disposez se mette à diverger avec le logiciel que vous êtes effectivement en train d'utiliser.
Enfin, il me semble plus pratique d'avoir un seul document à conserver plutôt que de devoir aller cliquer partout sur un site web le jour ou une information viendra à nous manquer.

Bien-sûr, le fait que la documentation soit un \emph{PDF} n'exclu pas le fait qu'on trouvera plus tard des ressources sur \emph{Floday} ailleurs, mais ce n'est pas en projet pour le moment.
Notez quand-même qu'il reste un vestige de wiki présent sur le bug-tracker~: \url{https://dev.spyzone.fr/floday/wiki}.
Notez également qu'il n'a pas été mit à jour depuis longtemps, et qu'une partie des informations qu'il contient sont fausse.
Vous ne devez donc pas le prendre comme référence.

\subsection{Le bug-tracker}
\existstill{1.0.0}
Un bug-tracker est disponible à l'adresse \url{dev.spyzone.fr/floday/tickets}.
Comme son nom l'indique, il liste l'intégralité des demandes d'évolutions et des rapport de bugs constatés sur l'intégralité du code de \emph{Floday}.

C'est un point d'entré privilégier si cette présente documentation ne répond pas correctement à l'une de vos question, si vous pensez avoir découvert une anomalie, ou si vous aimerez voir quelque chose de plus ou de différent dans l'application.
Avant de poster un nouveau ticket, il vous néanmoins demandé de vérifié si aucun doublon ne sera introduit par votre action.
En effet, limité le bruit et les redondances et un élément simple, à la porté de tous, et qui permet de faire gagner du temps précieux aux développeurs qui l'allouerons à meilleur esseyant ailleurs \smiley.

\subsection{Le dépôt \emph{Git}}
Certes, le projet est sensé être open-source, mais le dépôt \emph{Git} n'est pour l'instant pas totalement publique\dots{}
Les raisons derrières cette limitation sont en fait techniques~: actuellement, \emph{Gitolite} est utilisé pour faire la gestion des permissions sur le dépôt.
Or, ce dernié nécessite une connexion via \emph{SSH}, qui elle-même nécessite au serveur de posséder votre clef publique.
Sans cette clef, il est donc actuellement impossible de récupérer la moindre source.
Bien-sûr, je vous donnerez les accès en lecture si vous m'envoyez la votre par email.
Ce problème devrait néamoins être résolu un jour.
