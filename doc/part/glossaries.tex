\newglossaryentry{action}{
	name=action,
	plural=actions,
	description={%
		Une action consiste en un procédé actif réalisé par \emph{Floday} dans le cadre du déploiement.
		Elles peuvent être des exécutions de commandes, des écritures de configuration, ou de toute autres formes prévues par la définition du \gls{conteneur} en cours d'\gls{instantiation}.
	}
}

\newglossaryentry{application}{
	name=application,
	plural=applications,
	description={%
		Service concret rendu par le système informatique de référence, résultant du déploiement d'un \gls{conteneur}.
		C'est l'entité pour laquelle \emph{Floday} existe~: le processus final utilisé par les clients.
		Exemple~: un blog, un serveur \emph{Mumble}, un service de messagerie instantanée.
	}
}

\newglossaryentry{attribut}{
	name=attribut,
	plural=attributs,
	description={%
		L'attribut est une entité d'information permettant de définir un \gls{conteneur}.
		Il peut avoir des formes très variés, et se situent généralement au sein de la configuration d'un conteneur.
		Il ont également des formes hiérarchiques~: un attribut peut en contenir d'autres.
		De façon générale, les attributs peuplent intégralement les fichiers \emph{config.yml} présents au cœur de chaque conteneur.
	}
}

\newglossaryentry{conteneur}{
	name=conteneur,
	plural=conteneurs,
	description={%
		Le conteneur est une recette à appliquer pour produire une \gls{application}.
		Un même conteneur peut donc être utiliser pour générer plusieurs applications similères.
		Par exemple~: un conteneur \emph{Wordpress} peut être utiliser pour générer plusieurs blogs.
	}
}

\newglossaryentry{contraint}{
	name=contrainte -- de conteneur,
	plural=contraint,
	text=contraint,
	description={%
		Un \gls{conteneur} contraint est un conteneur dont son propre fonctionnement est conditionné par un autre, appelé conteneur \gls{gestionnaire}.
	}
}

\newglossaryentry{gestionnaire}{
	name=gestionnaire -- de conteneurs,
	text=gestionnaire,
	plural=gestionnaires,
	description={%
		Un \gls{conteneur} est gestionnaire s'il manage ou permet à d'autre conteneurs de fonctionner au travers lui.
		Un gestionnaire s'occupe de conteneurs \glspl{contraint}.
	}
}

\newglossaryentry{hote}{
	name=hôte,
	plural=hôtes,
	description={%
		L'hôte et le niveau zéro de l'\gls{imbrication}.
		Il s'agit du système physique sur lequels les \glspl{application} sont executées.
		Les \glspl{action} qui y sont effectues n'ont donc aucun confinement.
	}
}

\newglossaryentry{imbrication}{
	name=imbrication,
	description={%
		L'imbrication représente un niveau de \gls{contraint}.
		L'imbrication zéro représente l'\gls{hote} sur lequel les applications s'exécutes.
		Une application d'imbrication deux signifie qu'elle est contrainte par une autre application présente entre elle-même et l'hôte.
	}
}

\newglossaryentry{jeu_conteneur}{
	name=jeu de conteneurs,
	plural=jeux de conteneurs,
	description={%
		Le jeu de conteneurs constitue la définition même de ce que \emph{Floday} est capable de déployer.
		Par défaut, le jeu doit être présent dans le répertoire {\tt/etc/floday/containers}.
		Il est à la charge de l'utilisateur d'écrire son propre jeu, car seul lui est à-même de savoir comment son architecture devrait fonctionner.
	}
}

\newglossaryentry{param_applicatifs}{
	name=paramètre applicatif,
	plural=paramètres applicatifs,
	description={%
		Un paramètre applicatif est une entité de configuration appliquée à une \gls{application}.
		Il peut être implicite, si il est défini au niveau de la définition du \gls{conteneur} ou explicite, si il l'est plutôt au niveau du \gls{runfile}.
		Il peut aussi être redéfini à chaque niveau~: dans un conteneur hérité, fils puis dans le runfile.
		Comme exemple, nous pouvons prendre le cas d'un paramètre \emph{url} présent dans le conteneur \emph{Wordpress} permettant de savoir comment configurer correctement le \emph{CMS}.
	}
}

\newglossaryentry{runfile}{
	name=runfile,
	description={%
		Par défaut à l'emplacement {\tt/etc/runfile}, il s'agit du fichier définissant explicitement toutes les \glspl{application} a exécuter sur l'ensemble du «\,parc\,» informatique géré.
		Le même fichier est donc utilisé sur plusieurs hôtes pour permettre là encore de voir les interactions (procédés de backup, ou de répartion de la charge, etc.).
		Il contient donc une liste exaustive de toutes les applications à déployer, ainsi que la liste de tous les \glspl{param_applicatifs} non implicites.
	}
}

\newglossaryentry{instantiation}{
	name=instantiation,
	plural=instantiation,
	description={%
		Installation d'une \gls{application}.
		C'est-à-dire qu'on déroule tous les scripts d'exécution pour mettre en place l'application et la rendre utilsable.
	}
}
